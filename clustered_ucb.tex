% This is LLNCS.DEM the demonstration file of
% the LaTeX macro package from Springer-Verlag
% for Lecture Notes in Computer Science,
% version 2.4 for LaTeX2e as of 16. April 2010
%
\documentclass{llncs}
%
\usepackage{makeidx}  % allows for indexgeneration
\usepackage{macros}
%\usepackage{natbib}
%
%\usepackage[numbers]{natbib}
%% Hack natbib so it matches the LNCS style: reference list in a
%% section with small font and no square brackets.
%\renewcommand\bibsection
%  {\section*{\refname}\small\renewcommand\bibnumfmt[1]{##1.}}

\begin{document}
%
\frontmatter          % for the preliminaries
%
\pagestyle{headings}  % switches on printing of running heads
\addtocmark{Clustered UCB} % additional mark in the TOC

\mainmatter              % start of the contributions
%
\title{UCB with clustering and improved exploration}
%
\titlerunning{Clustered UCBV}  % abbreviated title (for running head)
%                                     also used for the TOC unless
%                                     \toctitle is used
%
\author{Subhojyoti Mukherjee${}^1$, L. A. Prashanth${}^1$, Nandan
Sudarsanam${}^2$, Balaraman Ravindran${}^1$}
%
\authorrunning{Subhojyoti Mukherjee et al.} % abbreviated author list (for running head)
%
%%%% list of authors for the TOC (use if author list has to be modified)
\tocauthor{Subhojyoti Mukherjee,  L. A. Prashanth, Nandan Sudarsanam and 
 Balaraman Ravindran}
%
\institute{${}^1$Department of Computer Science \& Engineering, 
${}^2$Department of Management Studies,\\ Indian Institute of
Technology Madras\\
%\email{subho@cse.iitm.ac.in}
}
%Universit\'{e} de Paris-Sud,
%Laboratoire d'Analyse Num\'{e}rique, B\^{a}timent 425,\\
%F-91405 Orsay Cedex, France
%,\\ WWW home page:
%\texttt{http://users/\homedir iekeland/web/welcome.html}

\maketitle              % typeset the title of the contribution

\begin{abstract}
In this paper, we present a novel algorithm for the stochastic multi-armed bandit (MAB) problem. Our proposed Efficient Clustered UCB method, referred to as EClusUCB partitions the arms into clusters and then follows the UCB-Improved strategy with aggressive exploration factors to eliminate sub-optimal arms, as well as entire clusters. Through a theoretical analysis, we establish that EClusUCB achieves a better gap-dependent regret upper bound than UCB-Improved~\cite{auer2010ucb} and MOSS~\cite{audibert2009minimax} algorithms. Further, numerical experiments on test-cases with small gaps between optimal and sub-optimal mean rewards show that EClusUCB results in lower cumulative regret than several popular UCB variants as well as MOSS, OCUCB~\cite{lattimore2015optimally}, Thompson sampling and Bayes-UCB\cite{kaufmann2012bayesian}. 
%We also present another algorithm called Adaptive Clustered UCB or AClusUCB which is intended to look at the effect of using more traditional approaches, like hierarchical clustering, for the grouping of arms.

\keywords{Multi-armed Bandits, Cumulative Regret, Clustering, UCB-Improved}
\end{abstract}

%\vspace*{-3em}
\section{Introduction}
\label{sec:intro}
\input{intro}

\section{Algorithm: Efficient Clustered UCB}
\label{sec:eclusucb}
\input{ealgo}

\section{Main results}
\label{sec:results}
	
We now state the main result that upper bounds the expected regret of EClusUCB.
\begin{theorem}[\textbf{\textit{Regret bound}}]
\label{Result:Theorem:1}
The regret $R_T$ of EClusUCB satisfies
\begin{align*}
&\E [R_{T}]\leq 
\sum\limits_{\substack{i\in A_{s^{*}},\\\Delta_{i} > b}}\bigg\lbrace \frac{C_1(\rho_{a})T^{1-\rho_{a}}}{\Delta_{i}^{4\rho_{a}-1}} + \Delta_{i}
+ \frac{32\log{(\psi T\frac{\Delta_{i}^{4}}{16})}}{\Delta_{i}} \bigg\rbrace
 + \! \! \sum\limits_{\substack{i\in A,\\\Delta_{i} > b}} \bigg\lbrace 2\Delta_{i}+
\frac{C_1(\rho_{s})T^{1-\rho_{s}}}{\Delta_{i}^{4\rho_{s}-1}} \\
%%%%%%%%%%%%%%%%%
&+ \frac{32\log{(\psi T\frac{\Delta_{i}^{4}}{16})}}{\Delta_{i}} 
+ \frac{32\log{(\psi T\frac{\Delta_{i}^{4}}{16})}}{\Delta_{i}}\bigg\rbrace 
+ \sum\limits_{\substack{i\in A_{s^{*}},\\ \Delta_{i} > b}} 
\frac{C_2(\rho_{a})T^{1-\rho_{a}}}{\Delta_{i}^{4\rho_{a}-1}}
+\sum\limits_{\substack{i\in A_{s^{*}},\\0 < \Delta_{i}\leq b}}\frac{C_2(\rho_a)T^{1-\rho_{a}}}{b^{4\rho_{a} -1}}\\ 
%%%%%%%
%&+ \!\sum\limits_{\substack{i\in A,\\ \Delta_{i} > b}}\! \! \frac{C_2(\rho_{s})T^{1-\rho_{s}}}{\Delta_{i}^{4\rho_{s}-1}}
% + \!\sum\limits_{\substack{i\in A,\\0 < \Delta_{i}\leq b}}\! \! \frac{C_2(\rho_{s})T^{1-\rho_{s}}}{b^{4\rho_{s} -1}}  \\
&+ \sum_{\substack{i\in A\setminus A_{s^*}:\\\Delta_{i}> b}}\frac{2C_{2}(\rho_{s})T^{1-\rho_{s}}}{\Delta_{i}^{4\rho_{s}-1}} +\sum_{\substack{i\in A \setminus A_{s^*}:\\ 0 < \Delta_{i} \leq b}}\frac{2C_{2}(\rho_{s})T^{1-\rho_{s}}}{b^{4\rho_{s}-1}}\\
& \!+\! \max\limits_{i:\Delta_{i}\leq b}\Delta_{i}T, 
\end{align*}
where $b\geq \sqrt{\frac{K}{14 T}}$, $C_1(x) = \frac{2^{1+4x}x^{2x}}{\psi^{x}}$, $C_2(x) = \frac{2^{2x+\frac{3}{2}}x^{2x}}{\psi^{x}}$, and $A_{s^{*}}$ is the subset of arms in cluster $s^{*}$ containing optimal arm $a^{*}$.
%, $\rho_{a}=\frac{1}{2},\rho_{s}=\frac{1}{2}$ and $\psi=K^{2}T$.
\end{theorem}
\begin{proof}
 See Section \ref{sec:proofTheorem}.
\end{proof}
We now specialize the result in the theorem above by substituting specific values for the exploration constants $\rho_{s}$, $\rho_{a}$ and $\psi$. 


%%%% Gap dependent bound
\begin{corollary}[\textbf{\textit{Gap-dependent bound}}]
\label{Result:Corollary:1}
With $\psi=\frac{T}{196\log (K)}$, $\rho_{a}=\frac{1}{2}$, and $\rho_{s}=\frac{1}{2}$,  we have the following gap-dependent bound for the regret of EClusUCB:
\begin{align*}
&\E [R_T] \!\le\! 
\sum_{\substack{i\in A_{s^{*}}:\\\Delta_{i} > b}}\bigg\lbrace \frac{192\sqrt{\log (K)}}{\Delta_{i}} + \Delta_{i} + 
 \frac{64\log{(T\frac{\Delta_{i}^{2}}{\sqrt{\log (K)}})}}{\Delta_{i}} \bigg\rbrace + \sum_{i\in A:\Delta_{i} > b}\bigg\lbrace\frac{112\sqrt{\log (K)}}{\Delta_{i}} \\
 %%%%%%%%%%%%%%%%%%%%%%
 & + 2\Delta_{i}  + \frac{128\log{(T\frac{\Delta_{i}^{2}}{\sqrt{\log (K)}})}}{\Delta_{i}}\bigg\rbrace
	  + \sum\limits_{\substack{i\in A_{s^{*}}:\\0< \Delta_{i} \leq b}}\frac{80\sqrt{\log (K)}}{\Delta_{i}}
	 + \sum\limits_{\substack{i\in A\setminus A_{s^{*}}:\\\Delta_{i} > b}}\frac{160\sqrt{\log (K)}}{\Delta_{i}} \\
	 %%%%%%%%%%%%%%%%%%%%
	 & + \sum\limits_{\substack{i\in A\setminus A \cup A_{s^{*}}:\\0 < \Delta_{i}\leq b}}\frac{160\sqrt{\log (K)}}{\Delta_{i}}  + \max\limits_{i\in A:\Delta_{i}\leq b}\Delta_{i}T, \quad \text{ for all }b\geq \sqrt{\frac{K}{14 T}}.
	\end{align*} 
\end{corollary}
\begin{proof}
 See Appendix \ref{App:Proof:Corollary:1}.
\end{proof}

The most significant term in the bound above is $\sum_{i\in A:\Delta_{i}\geq b}\frac{128\log{\big(T\frac{\Delta_{i}^{2}}{\sqrt{\log (K)}}\big)}}{\Delta_{i}}$ and hence, the regret upper bound for EClusUCB is of the order $O\bigg(\frac{K\log \big(\frac{T\Delta^{2}}{\sqrt{\log (K)}}\big)}{\Delta}\bigg)$. Since Corollary \ref{Result:Corollary:1} holds for all $\Delta \geq \sqrt{\frac{K}{14 T}} $, it can be clearly seen that for all $\sqrt{\frac{K}{14 T}} \leq \Delta\leq 1$ and $K\geq 2$, the gap-dependent bound is better than that of UCB1, UCB-Improved and MOSS (see Table~\ref{tab:regret-bds}). 

%Also, the gap-independent bound of UCB-Improved holds for $ \sqrt{\frac{K\log K}{T}} \Delta \leq 1$, 

%in ClusUCB if we take $\gamma$ such that $\frac{K}{\gamma}\approx e$ then $\Delta\geq\sqrt{\frac{K}{\gamma T}}\geq \sqrt{\frac{e}{T}}$ and we can easily mimic the same guarantee as UCB-Improved. %In the theoretical results as well as in the experiments we have taken $\gamma =7$. 
%In the experiments we have taken $\gamma =7$. 


\begin{corollary}[\textbf{\textit{Gap-independent bound}}]
\label{Result:Corollary:2}
Considering the same gap of $\Delta_{i} = \Delta =\sqrt{\frac{K\log K}{T}}$ for all ${i:i\neq *}$ and with $\psi=\frac{T}{196 \log K}$, $p=\left\lceil\frac{K}{\log K}\right\rceil$, $\rho_{a}=\frac{1}{2}$ and $\rho_{s}=\frac{1}{2}$, 
 we have the following gap-independent bound for the regret of EClusUCB:
\begin{align*}
 \E[R_{T}]\le & 540\frac{\sqrt{T}\log K}{\sqrt{K}} \!+\! \frac{64\sqrt{T\log K}\log{(\log K)}}{\sqrt{K}}\\
  &\!+\! 112\sqrt{KT} \!+\! 256\sqrt{KT\log K}
	 + \frac{128\sqrt{KT}\log{(\log K)}}{\sqrt{\log K}} + 300\sqrt{\frac{T\log K}{e}}\\
%%%%%%%%%%%%%%%%%%%
	& + 600\sqrt{\frac{T}{e}}(\log K)^{\frac{3}{2}} + 600 \frac{K}{K+\log K}\sqrt{KT}
\end{align*}
\end{corollary}
\begin{proof}
 See Appendix \ref{App:Proof:Corollary:2}.
\end{proof}


From the above result, we observe that the order of the regret upper bound of EClusUCB is $O(\sqrt{KT\log K})$, and this matches the order of UCB-Improved. However, this is not as low as the order $O(\sqrt{KT})$ of MOSS or OCUCB. Also, the gap-independent bound of UCB-Improved holds for $ \sqrt{\frac{e}{T}} \leq \Delta \leq 1$ while in our case the gap independent bound holds for $\sqrt{\frac{K}{14T}} \leq \Delta \leq 1$.

%\begin{corollary}[\textbf{\textit{Gap-independent bound}}]
%\label{Result:Corollary:3}
%With $\psi=K^{2}T$, $\rho_{a}=\frac{1}{4}$ ,$\rho_{s}=\frac{1}{2} $ and $b\approx\sqrt{\frac{K\log K}{T}}$, the regret of ClusUCB is bounded  by $\bigg\lbrace 2\sqrt{KT} + 64\sqrt{KT\log K} + \frac{32\log{(\log K)}}{\sqrt{\log K}} + 4\frac{\sqrt{KT}}{p}  + 16\sqrt{\frac{T}{K\log K}}\bigg\rbrace$.
%\end{corollary}
%\begin{proof}
% See Appendix \ref{App:Proof:Corollary:3}.
% 
%From the above result we see that this bound is less than than the bound in Corollary \ref{Result:Corollary:2}. Here we also define the error bound, which is the bound on the regret obtained after elimination of the optimal arm $a^{*}$. So the  error bound from Corollary \ref{Result:Corollary:2} is 
%
%\begin{align*}
%5.6\sqrt{KT} + 4\frac{\sqrt{KT}}{p}
%\end{align*}
%
%which is more than the error bound from Corollary \ref{Result:Corollary:3},
%
%\begin{align*}
%16\sqrt{\frac{T}{K\log K}} + 4\frac{\sqrt{KT}}{p}
%\end{align*}
%
%for $ \sqrt{\log K} \leq p\leq\frac{K}{2}$. So by taking $\rho_{a} < \rho_{s}$ we are able to reduce the error bound and this helps the algorithm to perform better in regimes where the gaps are small by keeping the optimal arm safe with high probability. 
%
%\end{proof}
%\vspace*{-1.5em}
\subsection*{Analysis of elimination error (Why Clustering?)}
%\vspace*{-0.4em}
Let $\widetilde R_T$ denote the contribution  to the expected regret in the case when the optimal arm $*$ gets eliminated during one of the rounds of EClusUCB. This can happen if a sub-optimal arm eliminates $*$ or if a sub-optimal cluster eliminates the cluster $s^*$ that contains $*$ -- these correspond to cases b2 and b3 in the proof of Theorem \ref{Result:Theorem:1} (see Section \ref{sec:proofTheorem}). 
We shall denote variant of EClusUCB that includes arm elimination condition only as EClusUCB-AE while EClusUCB corresponds to Algorithm \ref{alg:eclusucb}, which uses both arm and cluster elimination conditions. The regret upper bound for EClusUCB-AE is given in Proposition \ref{proofTheorem:Prop:1} in Appendix \ref{App:A}.

For EClusUCB-AE, the quantity $\widetilde R_T$ can be extracted from the proofs (in particular, case b2 in Appendix \ref{App:A}) and simplified using the values $\rho_{a}=\frac{1}{2}$ and $\psi=\frac{T}{196 \log K}$, to obtain $\widetilde R_T = 300\sqrt{KT\log K}+300\sqrt{KT}$. 
%A similar exercise for ClusUCB-CE (see Case b2 in Appendix \ref{App:B}) with $\rho_{s}=\frac{1}{2}$ and $\psi=\frac{T}{\log K}$ yields $\tilde R_T = 4\sqrt{KT\log K}$. 
Finally, for EClusUCB, the relevant terms from Theorem \ref{Result:Theorem:1} that corresponds to $\widetilde R_T$ can be simplified with $\rho_{a}=\frac{1}{2}$, $\rho_{s}=\frac{1}{2},p=\big\lceil \frac{K}{\log K} \big\rceil$ and $\psi=\frac{T}{196\log K}$ (as in Corollary \ref{Result:Corollary:2} to obtain  
$\tilde R_T = \frac{300 \sqrt{T}\log K^{\frac{3}{2}} }{\sqrt{K}} + \frac{300 \sqrt{T}\log K}{\sqrt{K}} + 600 \frac{K}{K+\log K}\sqrt{KT\log K} + 600 \frac{K}{K+\log K}\sqrt{KT}$. Hence, in comparison to EClusUCB-AE which has an elimination regret bound of $O(\sqrt{KT\log K})$, the elimination error regret bound of EClusUCB is lower and of the order $O(\frac{K}{K+\log K}\sqrt{KT\log K})$. Thus, we observe that clustering in conjunction with improved exploration via $\rho_{a},\rho_{s}$,$p$ and $\psi$ helps in reducing the factor associated with $\sqrt{KT\log K}$ for the gap-independent error regret bound for EClusUCB. Also in section \ref{sec:expts}, in experiment $4$ we show that EClusUCB outperforms EClusUCB-AE. A table containing the regret error bound is shown in Appendix \ref{App:E}.
% and further experiment showing that the performance of EClusUCB against CCB\cite{liu2016modification} is shown in Appendix \ref{App:MoreExp}.
%showing that the performance of EClusUCB-AE is indeed sub-optimal and choice of $p$ is indeed close to optimal

Finally, the simple regret guarantee of EClusUCB is weaker than CCB\cite{liu2016modification} which is shown in Theorem \ref{Result:Theorem:3} and Corollary \ref{Result:Corollary:3} in Appendix \ref{App:SR_EClusUCB}. But, this is expected as EClusUCB is geared towards minimizing cumulative regret whereas CCB is made for minimizing simple regret. Also we know from \cite{bubeck2009pure} that algorithms that tend to minimize cumulative regret necessarily ends up having a poorer simple regret guarantee. 
%The regret upper bound for EClusUCB-AE is given in Proposition \ref{proofTheorem:Prop:1} in Appendix \ref{App:A}. 



\section{Proof of Theorem 1}
\label{sec:proofTheorem}
\begin{proof}

%$\Delta_{i}^{'}=r_{a_{\max_{s_{k}}}} - r_{i}$ such that $a_{i}\in s_{k}$,
% m_{i}^{'}=\min{\lbrace m|\sqrt{\rho_{a}\epsilon_{m}} < \frac{\Delta_{i}^{'}}{2} \rbrace}
Let $A^{'}=\lbrace i \in A,\Delta_{i}> b\rbrace$,  $A^{''}=\lbrace i \in A, \Delta_{i} > 0\rbrace$, $A^{'}_{s_{k}}=\lbrace i \in A_{s_{k}},\Delta_{i}> b\rbrace$ and $A^{''}_{s_{k}}=\lbrace i \in A_{s_{k}}, \Delta_{i} > 0 \rbrace$. $C_{g}$ is the cluster set containing max payoff arm from each cluster in $g$-th round. The arm having the true highest payoff in a cluster $s_{k}$ is denote by $a_{\max_{s_{k}}}$. Let for each sub-optimal arm ${i}\in A$, $m_{i}=\min{\lbrace m|\sqrt{\epsilon_{m}} < \frac{\Delta_{i}}{2} \rbrace}$ and let for each cluster $s_{k}\in S$, $g_{s_{k}}=\min{\lbrace g|\sqrt{\epsilon_{g}} < \frac{\Delta_{a_{\max_{s_{k}}}}}{2} \rbrace}$. 
Let $\check{A}=\lbrace {i}\in A^{'} | {i}\in s_{k} , \forall s_{k}\in S \rbrace$. Also $z_{i}$ denotes total number of times an arm $i$ has been pulled. In the $m$-th round, $n_{m}$ denotes the number of pulls allocated to the surviving arms in $B_{m}$. 
The analysis proceeds by considering the contribution to the regret in each of the following cases:

\textbf{Case a:} \textit{Some sub-optimal arm ${i}$ is not eliminated in round $\max(m_{i},g_{s_{k}})$ or before, with the optimal arm ${*}\in C_{\max(m_{i},g_{s_{k}})}$.}
We consider an arbitrary sub-optimal arm ${i}$ and analyze the contribution to the regret when $i$ is not eliminated in the following exhaustive sub-cases:\\
\textbf{Case a1:} \textit{In round $\max(m_{i},g_{s_{k}})$, ${i} \in s^{*}$.}

Similar to case (a) of \cite{auer2010ucb}, observe that when the following two conditions hold, arm $i$ gets eliminated:
\begin{align}
\hat{r}_{i}  \le r_{i} + c_{m_i} \text{ and } 
 \hat{r}^{*}\geq  r^{*} - c_{m_i}, \label{eq:armelim-casea1}
\end{align}
where  $c_{m_i} = \sqrt{\frac{\rho_{a}\log (\psi T\epsilon_{m_{i}}^{2})}{2 n_{m_i}}}$. 
%Note that any arm $i$ in the $m_{i}$-th round cannot be pulled more than $n_{m_{i}}$ number of times as it will get eliminated as arm elimination condition is being checked in every timestep. 
As arm elimination condition is being checked in every timestep, for $z_i= n_{m_i}$, the arm $i$ gets eliminated because 
  \begin{align*}
\hat{r}_{i} +c_{m_i} & \leq r_{i} + 2c_{m_i} < r_{i} + \Delta_{i} - 2c_{m_i}\\
 &\leq r^{*} -2c_{m_i} \leq \hat{r}^{*} - c_{m_i}  .
  \end{align*}
%\todos[inline]{The stmt ``bound the probability of the event $\hat{r}_{i}+c_{m_{i}}\leq \hat{r}^{*}-c_{m_{i}}$'' is wrong. We bound the complementary event using Hoeffding} 
%\todos[inline]{The stmt ``$m_{i}$ does not happen'' makes no sense given that we are in round $\max(m_i,g_i)$}
%\todos[inline]{What is $c_m$?}
  %Now, $c_{m_{i}}=\sqrt{\frac{\rho_{a}\log (\psi T\epsilon_{m_{i}}^{2})}{2 n_{m_{i}}}}$.
  %$n_{m_i} \leq n_{m_*}$ (as $*\in B_{m_i}$) and so
In the above, we have used the fact that\\ $ c_{m_i} = \sqrt{\epsilon_{m_{i}+1}} < \frac{\Delta_{i}}{4}$, since $n_{m_{i}}=\frac{\log{(\psi T\epsilon_{m_{i}}^{2})}}{2\epsilon_{m_{i}}}$ and $\rho_{a}\in (0,\frac{1}{2}]$. From the foregoing, we have to bound the events complementary to that in \eqref{eq:armelim-casea1} for an arm $i$ to not get eliminated. Considering Chernoff-Hoeffding bound this is done as follows:
  \begin{align*}
&\mathbb{P}\left(\hat{r}_{i}\geq r_{i} + c_{m_i}\right)\leq \exp(-2c_{m_i}^{2}n_{m_i})\\
&\leq \exp(-2 * \frac{\rho_{a}\log (\psi T\epsilon_{m_{i}}^{2})}{2 n_{m_i}} *n_{m_i})
\leq \frac{1}{(\psi T\epsilon_{m_{i}}^{2})^{\rho_{a}}}   
  \end{align*}
Along similar lines, we have 
$\mathbb{P}\left(\hat{r}^{*}\leq r^{*} - c_{m_i}\right)\leq \frac{1}{(\psi  T\epsilon_{m_{i}}^{2})^{\rho_{a}}}.$ Thus, the probability that a sub-optimal arm ${i}$ is not eliminated in any round on or before $m_{i}$ is bounded above by  $\bigg(\frac{2}{(\psi T\epsilon_{m_{i}}^{2})^{\rho_{a}}}\bigg)$. 
 Summing up over all arms in $A_{s^{*}}^{'}$ in conjunction with a simple bound of $T\Delta_{i}$ for each arm, and noting that $C_1(x) = \frac{2^{1+4x}}{\psi^{x}}$ we obtain
   \begin{align*}
&\sum_{i\in A_{s^{*}}^{'}}\bigg(\dfrac{2T\Delta_{i}}{(\psi T\epsilon_{m_{i}}^{2})^{\rho_{a}}}\bigg)
\leq\sum_{i\in A_{s^{*}}^{'}}\bigg(\frac{2T\Delta_{i}}{(\psi T\dfrac{\Delta_{i}^{4}}{16})^{\rho_{a}}}\bigg)
=\sum_{i\in A_{s^{*}}^{'}}\bigg(\frac{C_{1}(\rho_{a})T^{1-\rho_{a}}}{\Delta_{i}^{4\rho_{a}-1}}\bigg) %\text{, where } C_1(x) = \frac{2^{1+4x}}{\psi^{x}}
   \end{align*}

%%%%%%%%%%%%%%%%%%%%%%%%%%%%%%%%%%%%%%%%%%%%%%%%%%%%%%%%%%%%%%%%%%%%%%%%%%%%%%%%%%%%%%%%%%%%%%%   



%%%%%%%%%%%%%%%%%%%%%%%%%%%%%%%%%%%%%%%%%%%%%%%%%%%%%%%%%%%%%%%%%%%%%%%%%%%%%%%%%%%%%%%%%%%%%%%   
\textbf{Case a2:} \textit{In round $\max(m_{i},g_{s_{k}})$, ${i} \in s_k$ for some $s_k \ne s^{*}$.}

Following a parallel argument like in Case $a1$, as cluster elimination condition is being checked at  every timestep, we have to bound the following two events of arm $a_{\max_{s_{k}}}$ not getting eliminated on or before $g_{s_{k}}$-th round,
\begin{align*}
  \hat{r}_{a_{\max_{s_{k}}}} \geq r_{a_{\max_{s_{k}}}} +c_{g_{s_{k}}} \text{ and } \hat{r}^{*} \leq r^{*} - c_{g_{s_{k}}} 
\end{align*} 

We can prove using Chernoff-Hoeffding bounds and considering independence of events mentioned above, that for $c_{g_{s_{k}}}=\sqrt{\frac{\rho_{s} \log (\psi T\epsilon_{g_{s_{k}}}^{2})}{2 n_{g_{s_{k}}}}}$ and  $z_{a_{\max_{s_{k}}}}=n_{g_{s_{k}}}=\frac{\log{(\psi T\epsilon_{g_{s_{k}}}^{2})}}{2\epsilon_{g_{s_{k}}}}$ the probability of the above two events is bounded by $\bigg(\dfrac{2}{(\psi  T\epsilon_{g_{s_{k}}}^{2})^{\rho_{s}}}\bigg)$.
%Summing, the two up, the probability that a sub-optimal cluster arm $a_{\max_{s_{k}}}\in C_{g_{s_{k}}}$ is not eliminated
  Now, for any round $g_{s_{k}}$, all the elements of $C_{\max(m_{i},g_{s_{k}})}$ are the respective maximum payoff arms of their cluster $s_{k}, \forall s_{k}\in S$, and since clusters are fixed so we can bound the maximum probability that a sub-optimal arm ${i}\in A^{'}$  and ${i}\in s_{k}$ such that $a_{\max_{s_{k}}}\in C_{g_{s_{k}}}$ is not eliminated on or before the $g_{s_{k}}$-th round by the same probability as above. 
%\begin{align*}
%\bigg(\frac{2}{(\psi T\epsilon_{g_{s_{k}}}^{2})^{\rho_{s}}}\bigg)
%\end{align*} 
%Summing up over all arms in $s_{k}$ and bounding trivially by $T\Delta_{i}$,
%\begin{align*}
%\sum_{i\in A_{s_{k}}}\bigg(\frac{2T\Delta_{i}}{(\psi T\epsilon_{g_{s_{k}}}^{2})^{\rho_{s}}}\bigg)
%\end{align*}
Summing up over all $p$ clusters and bounding the regret for each arm $i\in A_{s_{k}}^{'}$ trivially by $T\Delta_{i}$,
 \begin{align*}
 &\sum_{k=1}^{p}\sum_{i\in A_{s_{k}}^{'}}\bigg(\frac{2T\Delta_{i}}{(\psi T\frac{\Delta_{i}^{4}}{16})^{\rho_{s}}}\bigg) = \sum_{i\in A^{'}}\bigg(\frac{2T\Delta_{i}}{(\psi  T\frac{\Delta_{i}^{4}}{16})^{\rho_{s}}}\bigg) \\
 &\leq \sum_{i\in A^{'}}\bigg(\frac{2^{1+4\rho_{s}}T^{1-\rho_{s}}}{\psi^{\rho_{s}}\Delta_{i}^{4\rho_{s}-1}}\bigg) = \sum_{i\in A^{'}}\frac{C_{1}(\rho_{s})T^{1-\rho_{s}}}{\Delta_{i}^{4\rho_{s}-1}}
 \end{align*}
% &= \sum_{i\in A^{'}}\bigg(\frac{C_{1}(\rho_{s})T^{1-\rho_{s}}}{\Delta_{i}^{4\rho_{s}-1}}\bigg) \text{, where } C_1(x) = \frac{2^{1+4x}x^{2x}}{\psi^{x}}
%&\leq \sum_{i\in A}\bigg(\frac{2^{1+4\rho_{s}}T^{1-\rho_{s}}\rho_{s}^{2\rho_{s}}\Delta_{i}}{\psi^{\rho_{s}}\Delta_{i}^{4\rho_{s}}}\bigg)\\

Summing the bounds in Cases $a1-a2$ and observing that the bounds in the aforementioned cases hold for any round $C_{\max \lbrace m_i,g_{s_k}\rbrace}$, we obtain the following contribution to the expected regret from case a:
   %Taking summation of the events mentioned above($a1$-$a4$) gives us an upper bound on the regret given that the optimal arm $a^{*}$ is still surviving, 
\begin{align*}
&\sum_{i\in A_{s^*}} \frac{C_{1}(\rho_{a})T^{1-\rho_{a}}}{\Delta_{i}^{4\rho_{a}-1}} + \sum_{i\in A^{'}}\bigg(\frac{C_{1}(\rho_{s})T^{1-\rho_{s}}}{\Delta_{i}^{4\rho_{s}-1}}\bigg)
\end{align*}

%%%%%%%%%%%%%%%%%%%%%%%%%%%%%%%%%%%%%%%%%%%%%%%%%%%%%%%%%%%%%%%%%%%%%%%%%%%%%%%%%%%%%%%%%%%%
\textbf{Case b:} \textit{For each arm $i$, either ${i}$ is eliminated in round $\max (m_{i},g_{s_{k}})$ or before or there is no optimal arm ${*}$ in $C_{\max(m_{i},g_{s_{k}})}$.} \\
\textbf{Case b1:} \textit{${*}\in C_{\max(m_{i},g_{s_{k}})}$ for each arm $i \in A'$ and cluster $s_k \in \check A$.} 
The condition in the case description above implies the following: \\
\begin{inparaenum}[\bfseries (i)]
\item each sub-optimal arm ${i}\in A^{'}$ is  eliminated on or before $\max (m_{i},g_{s_{k}})$ and hence  pulled not more than $z_i < n_{m_{i}}$ number of times.\\
\item each sub-optimal cluster $s_k \in \check A$ is  eliminated on or before $\max (m_{i},g_{s_{k}})$ and hence  pulled not more than $z_{a_{\max_{s_{k}}}} < n_{g_{s_{k}}}$ number of times.
\end{inparaenum}

Hence, the maximum regret suffered due to pulling of a sub-optimal arm or a sub-optimal cluster is no more than the following:
 \begin{align*}
 &\sum_{i\in A^{'}}\Delta_{i}\bigg\lceil\dfrac{\log{(\psi T\epsilon_{m_{i}}^{2})}}{2\epsilon_{m_{i}}}\bigg\rceil 
\!+\! \sum_{k=1}^{p}\sum_{i\in A_{s_{k}}^{'}}\Delta_{i}\bigg\lceil\dfrac{\log{(\psi T\epsilon_{g_{s_{k}}}^{2})}}{2\epsilon_{g_{s_{k}}}}\bigg\rceil \\
%%%%%%%%%%%%%%%%%%%%
&\leq\sum_{i\in A^{'}}\Delta_{i}\bigg(1+\dfrac{32\log{\left(\psi T\left(\frac{\Delta_{i}}{2}\right)^{4}\right)}}{\Delta_{i}^{2}}\bigg) 
\quad+ \sum_{i\in A^{'}}\Delta_{i}\bigg(1+\dfrac{32\log{\left(\psi T\left(\frac{\Delta_{i}}{2}\right)^{4}\right)}}{\Delta_{i}^{2}}\bigg)
\\
%%%%%%%%%%%%%%%%%%%%
 &\leq \sum_{i\in A^{'}}\!\bigg[ 2\Delta_{i}+\dfrac{32(\log{(\psi T\dfrac{\Delta_{i}^{4}}{16})} + \log{(\psi T\dfrac{\Delta_{i}^{4}}{16})})}{\Delta_{i}} \bigg]
%  \\
% & \qquad \qquad +\dfrac{32\rho_{s}\log{(\psi T\dfrac{\Delta_{i}^{4}}{16\rho_{s}^{2}})}}{\Delta_{i}}\bigg\rbrace 
 \end{align*}
In the above, the first inequality follows since $\sqrt{\epsilon_{m_{i}}} < \frac{\Delta_{i}}{2}$ and $\sqrt{\epsilon_{n_{g_{s_{k}}}}} < \frac{\Delta_{a_{\max_{s_{k}}}}}{2}$.

%&\leq\Delta_{i}\bigg(1+\dfrac{32\rho_{a}\log{(\psi T\dfrac{\Delta_{i}^{4}}{16\rho_{a}^{2}})}}{\Delta_{i}^{2}}\bigg)\\
%&\leq\Delta_{i}\bigg\lceil\dfrac{2\log{(\psi T(\dfrac{\Delta_{i}}{2\sqrt{\rho_{a}})^{4})}}}{(\dfrac{\Delta_{i}}{2\sqrt{\rho_{a}}})^{2}}\bigg\rceil \\
%\text{, since } \sqrt{\rho_{a}\epsilon_{m_{i}}}\leq\dfrac{\Delta_{i}}{2}
 
%%%%%%%%%%%%%%%%%%%%%%%%%%%%%%%%%%%%%%%%%%%%%%%%%%%%%%%%%%%%%%%%%%%%%%%%%%%%%%%%%%%%%%%%%%%%%%%   
%\textbf{Case b2:} \textit{Optimal arm $a^{*}$ is eliminated by a sub-optimal arm.}\\
  %
	%This, can happen in $3$ ways,
%\newline
\textbf{Case b2:} \textit{${*}$ is eliminated by some sub-optimal arm in $s^*$} \\
%In this case, we are concerned with the arm elimination condition only. 
Optimal arm $*$ can get eliminated by some sub-optimal arm $i$ only if arm elimination condition holds, i.e., 
\begin{align*}
\hat r_{i} - c_{m_i} > \hat{r}^{*}+ c_{m_i},
\end{align*}
where, as mentioned before, $c_{m_i}  =\sqrt{\frac{\rho_{a}\log (\psi T\epsilon_{m_{i}}^{2})}{2 n_{m_i}}}$.
From analysis in Case $a1$, notice that, if \eqref{eq:armelim-casea1} holds in conjunction with the above, arm $i$ gets eliminated. Also, recall from Case $a1$ that the events complementary to \eqref{eq:armelim-casea1} have low-probability and can be upper bounded by $\frac{2}{(\psi  T\epsilon_{m_{*}}^{2})^{\rho_{a}}}$. Moreover, a sub-optimal arm that eliminates $*$ has to survive until round $m_*$. In other words, 
all arms ${j}\in s^{*}$ such that $m_{j} < m_{*}$ are eliminated on or before $m_*$ (this corresponds to case b1). 
Let, the arms surviving till $m_{*}$ round be denoted by $A^{'}_{s^{*}}$. This leaves any arm $a_{b}$ such that $m_{b}\geq m_{*} $ to still survive and eliminate arm ${*}$ in round $m_{*}$. Let, such arms that survive ${*}$ belong to $A^{''}_{s^{*}}$. Also maximal regret per step after eliminating ${*}$ is the maximal $\Delta_{j}$ among the remaining arms in $A^{''}_{s^{*}}$ with $m_{j}\geq m_{*}$.  Let $m_{b}=\min\lbrace m|\sqrt{\epsilon_{m}}<\frac{\Delta_{b}}{2}\rbrace$. Let $C_2(x) = \frac{2^{2x+\frac{3}{2}}}{\psi^{x}}$. Hence, the maximal regret after eliminating the arm ${*}$ is upper bounded by, 
\begin{align*}
&\sum_{m_{*}=0}^{max_{j\in A^{'}_{s^{*}}}m_{j}}\sum_{\substack{i\in A^{''}_{s^{*}}: \\ m_{i}\geq m_{*}}}\bigg(\dfrac{2}{(\psi  T\epsilon_{m_{*}}^{2})^{\rho_{a}}} \bigg).T\max_{\substack{j\in A^{''}_{s^{*}}: \\ m_{j}\geq m_{*}}}{\Delta}_{j}\\
%%%%%%%%%%%%%%%
&\leq\sum_{m_{*}=0}^{max_{j\in A^{'}_{s^{*}}}m_{j}}\sum_{i\in A^{''}_{s^{*}}:m_{i} \geq m_{*}}\bigg(\dfrac{2}{(\psi  T\epsilon_{m_{*}}^{2})^{\rho_{a}}} \bigg).T.2\sqrt{\epsilon_{m_{*}}} \\
%%%%%%%%%%%%%%%
&\leq\sum_{m_{*}=0}^{max_{j\in A^{'}_{s^{*}}}m_{j}}\sum_{i\in A^{''}_{s^{*}}:m_{i} \geq m_{*}}4\bigg(\dfrac{T^{1-\rho_{a}}}{\psi^{\rho_{a}}\epsilon_{m_{*}}^{2\rho_{a}-\frac{1}{2}}} \bigg)\\
&\leq\sum_{i\in A^{''}_{s^{*}}:m_{i} \geq m_{*}}\sum_{m_{*}=0}^{\min{\lbrace m_{i},m_{b}\rbrace}}\bigg(\dfrac{4T^{1-\rho_{a}}}{\psi^{\rho_{a}}2^{-(2\rho_{a}-\frac{1}{2})m_{*}}} \bigg)\\
%%%%%%%%%%%%%%
&\!\leq\!\!\sum_{i\in A^{'}_{s^{*}}}\frac{4T^{1-\rho_{a}}}{\psi^{\rho_{a}}2^{-(2\rho_{a}-\frac{1}{2})m_{*}}}\! +\!\!\!\sum_{i\in A^{''}_{s^{*}}\setminus A^{'}_{s^{*}}}\!\frac{4T^{1-\rho_{a}}}{\psi^{\rho_{a}}2^{-(2\rho_{a}-\frac{1}{2})m_{b}}} \\
%%%%%%%%%%%%%%
&\!\leq\!\!\sum_{i\in A^{'}_{s^{*}}}\frac{T^{1-\rho_{a}}2^{2\rho_{a}+\frac{3}{2}}}{\psi^{\rho_{a}}\Delta_{i}^{4\rho_{a}-1}} \!+\!\!\!\sum_{i\in A^{''}_{s^{*}}\setminus A^{'}_{s^{*}}}\!\!\frac{T^{1-\rho_{a}}2^{2\rho_{a}+\frac{3}{2}}}{\psi^{\rho_{a}}b^{4\rho_{a}-1}} \\
%%%%%%%%%%%%%%
& = \sum_{i\in A^{'}_{s^{*}}}\dfrac{ C_{2}(\rho_{a}) T^{1-\rho_{a}}}{\Delta_{i}^{4\rho_{a}-1}} +\sum_{i\in A^{''}_{s^{*}}\setminus A^{'}_{s^{*}}}\dfrac{C_{2(\rho_{a})}T^{1-\rho_{a}}}{b^{4\rho_{a}-1}}.
\end{align*}

%&\text{ since } \sqrt{\rho_{a}\epsilon_{m}}<\dfrac{\Delta_{i}}{2}\\
%&\leq\sum_{i\in A^{'}}\dfrac{4\rho_{a}^{2\rho_{a}}T^{1-\rho_{a}}*2^{2\rho_{a}-\frac{1}{2}}}{\psi^{\rho_{a}}\Delta_{i}^{4\rho_{a}-1}} +\sum_{i\in A^{''}\setminus A^{'}}\dfrac{4\rho_{a}^{2\rho_{a}}T^{1-\rho_{a}}*2^{2\rho_{a}-\frac{1}{2}}}{\psi^{\rho_{a}}b^{4\rho_{a}-1}} \\

% \begin{align*}
% &\sum_{i\in A^{'}_{s^{*}}}\bigg(\dfrac{C_{2}(\rho_{a})T^{1-\rho_{a}}}{\Delta_{i}^{4\rho_{a} -1}} \bigg)+\sum_{i\in A^{''}_{s^{*}}\setminus A^{'}_{s^{*}}}\bigg(\dfrac{C_{2}(\rho_{a})T^{1-\rho_{a}}}{b^{4\rho_{a} -1}} \bigg)
% \end{align*}
%We also see that here, we are concerned only within $s^{*}$ because of our assumption that there is only one $a^{*}\in A$ and clusters are fixed.


%%%%%%%%%%%%%%%%%%%%%%%%%%%%%%%%%%%%%%%%%%%%%%%%%%%%%%%%%%%%%%%%%%%%%%%%%%%%%%%%%%%%%%%%%%%%%%%   
\textbf{Case b3:} \textit{$s^{*}$ is eliminated by some sub-optimal cluster.} 
Let $C_{g}^{'}=\lbrace a_{max_{s_{k}}}\in A^{'}|\forall s_{k}\in S \rbrace$ and $C_{g}^{''}=\lbrace a_{max_{s_{k}}}\in A^{''}|\forall s_{k}\in S \rbrace$. A sub-optimal cluster $s_k$ will eliminate $s^*$ in round $g_*$ only if the cluster elimination condition of Algorithm \ref{alg:eclusucb} holds, which is the following when ${*}\in C_{g_{*}}$:
\begin{align}
\hat r_{a_{\max_{s_k}}} - c_{g_*} > \hat{r}^{*}+ c_{g_*}.
\label{eq:caseb3-cluselim}
\end{align}
Notice that when ${*}\notin C_{g_{*}}$, since $r_{a_{max_{s_{k}}}}>r^{*}$, the inequality in \eqref{eq:caseb3-cluselim} has to hold for cluster $s_k$ to eliminate $s^*$.
As in case $b2$, the probability that a given sub-optimal cluster $s_k$ eliminates $s^*$ is upper bounded by  $\frac{2}{(\psi T\epsilon_{g_{s^{*}}}^{2})^{\rho_{s}}}$ and all sub-optimal clusters with $g_{s_{j}}< g_{*}$ are eliminated before round $g_*$. This leaves any arm $a_{\max_{s_{b}}}$ such that $g_{s_{b}}\geq g_{*}$ to still survive and eliminate arm ${*}$ in round $g_{*}$. Let, such arms that survive ${*}$ belong to $C_{g}^{''}$. Hence, following the same way as case $b2$,  the maximal regret after eliminating ${*}$ is,
 \begin{align*}
 \!\!\sum_{g_{*}=0}^{\max\limits_{a_{\max_{s_{j}}}\in C_{g}^{'}}g_{s_{j}}}\!\!\!\!\!\sum_{\substack{\scriptsize a_{\max_{s_{k}}}\in C_{g}^{''}: \\ g_{s_{k}} \geq g_{*}}}\bigg(\dfrac{2}{(\psi T\epsilon_{g_{s^{*}}}^{2})^{\rho_{s}}} \bigg)T\max_{\substack{a_{\max_{s_{j}}}\in C_{g}^{''}: \\ g_{s_{j}}\geq g_{*}}}{\Delta}_{a_{\max_{s_{j}}}}
 \end{align*}
Using $A'\supset C_{g}^{'}$ and $A''\supset C_{g}^{''}$, we can bound the regret contribution from this case in a similar manner as Case $b2$ as follows:
% \begin{align*}
%  & \sum_{g_{*}=0}^{max_{j\in A^{'}}g_{s_{j}}}\sum_{\substack{i\in A^{''}: \\ g_{s_{k}}\geq g_{*}}}\bigg(\dfrac{2}{(\psi T\epsilon_{g_{s^{*}}}^{2})^{\rho_{s}}} \bigg).T\max_{\substack{j\in A^{''}: \\ g_{s_{j}}\geq g_{*}}}{\Delta}_{a_{\max_{s_{j}}}}
% \end{align*}
% Like Case $b2$, we can bound the regret as,
\begin{align*}
 &\!\!\sum_{i\in A^{'}\setminus A_{s^*}^{'}}\frac{T^{1-\rho_{s}}2^{2\rho_{s}+\frac{3}{2}}}{\psi^{\rho_{s}}\Delta_{i}^{4\rho_{s}-1}} 
 \!+\!\!\!\sum_{i\in A^{''}\setminus A^{'}\cup A_{s^*}^{'}}\!\!\!\!\frac{T^{1-\rho_{s}}2^{2\rho_{s}+\frac{3}{2}}}{\psi^{\rho_{s}}b^{4\rho_{s}-1}} \\
 & = \sum_{i\in A^{'}\setminus A_{s^*}^{'}}\frac{C_{2}(\rho_{s})T^{1-\rho_{s}}}{\Delta_{i}^{4\rho_{s}-1}} +\sum_{i\in A^{''}\setminus A^{'}\cup A_{s^*}^{'}}\frac{C_{2}(\rho_{s})T^{1-\rho_{s}}}{b^{4\rho_{s}-1}} 
\end{align*}

%%%%%%%%%%%%%%%%%%%%%%%%%%%%%%%%%%%%%%%%%%%%%%%%%%%%%%%%%%%%%%%%%%%%%%%%%%%%%%%%%%%%%%%%%
\textbf{Case b4:} \textit{${*}$ is not in $C_{\max(m_{i},g_{s_{k}})}$, but belongs to $B_{\max(m_{i},g_{s_{k}})}$.}

In this case the optimal arm ${*}\in s^{*}$ is not eliminated, also $s^{*}$ is not eliminated. So, for all sub-optimal arms $i$ in $A_{s^*}^{'}$ which gets eliminated on or before $\max \lbrace m_{i},g_{s_{k}} \rbrace$ will get pulled no less than $z_i < \left\lceil\dfrac{2\log{(\psi T\epsilon_{m_{i}}^{2})}}{\epsilon_{m_{i}}}\right\rceil$ number of times, which leads to the following bound the contribution to the expected regret, as in Case $b1$:
\begin{align*}
 &\sum_{i\in A_{s^*}^{'}}\bigg\lbrace \Delta_{i}+\dfrac{32\log{(\psi T\dfrac{\Delta_{i}^{4}}{16})}}{\Delta_{i}} \bigg\rbrace 
\end{align*} 

For arms $a_i \notin s^*$, the contribution to the regret cannot be greater than that in Case $b3$. So the regret is bounded by,

\begin{align*}
\sum_{i\in A^{'}\setminus A_{s^*}^{'}}\dfrac{C_{2}(\rho_{s})T^{1-\rho_{s}}}{\Delta_{i}^{4\rho_{s}-1}} +\sum_{i\in A^{''}\setminus A^{'} \cup A_{s^*}^{'}}\dfrac{C_{2}(\rho_{s})T^{1-\rho_{s}}}{b^{4\rho_{s}-1}}
\end{align*}
The main claim follows by summing the contributions to the expected regret from each of the cases above.
\end{proof}




%\begin{proposition}
%\label{proofTheorem:Prop:1}
%The regret $R_T$ for ClusUCB-AE satisfies
%\begin{align*}
%&\E [R_{T}]\leq \sum\limits_{\substack{i\in A\\\Delta_{i} > b}}\bigg\lbrace\frac{C_{1}(\rho_{a})T^{1-\rho_{a}}}{\Delta_{i}^{4\rho_{a}-1}} + \Delta_{i}+\frac{32\rho_{a}\log{(\frac{\psi  T\Delta_{i}^{4}}{16\rho_{a}^{2}})}}{\Delta_{i}}\\
%& +  \frac{C_{2}(\rho_{a})T^{1-\rho_{a}}}{\Delta_{i}^{4\rho_{a} -1}}  \bigg \rbrace +\sum\limits_{\substack{i\in A\\0 <\Delta_{i}\leq b}}\frac{C_{2}(\rho_{a})T^{1-\rho_{a}}}{b^{4\rho_{a} -1}}  + \max_{\substack{i\in A: \\ \Delta_{i}\leq b}}\Delta_{i}T,
%\end{align*}
%for all $b\geq\sqrt{\frac{K}{14 T}}$. In the above, $C_1, C_2$ are as defined in Theorem \ref{Result:Theorem:1}.
%% (x) = \frac{2^{1+4x}x^{2x}}{\psi^{x}}$,  $C_2(x) = \frac{2^{2x+\frac{3}{2}}x^{2x}}{\psi^{x}}$, $\rho_{a}=\frac{1}{2}$ is the arm elimination parameter, $\psi=K^{2}T$ is the exploration regulatory factor, $p$ is the number of clusters and $T$ is the horizon.
%\end{proposition}
%\begin{proof}
%See Appendix \ref{App:A}.
%\end{proof}



%%%%%%%%%%%%%%%%%%%%%%%%%
%Not giving ClusUCB-CE	
%%%%%%%%%%%%%%%%%%%%%%%%%

%\begin{proposition}
%\label{proofTheorem:Prop:2}
%The regret $R_T$ for ClusUCB-CE satisfies,
%\begin{align*}
%&\E [R_{T}]\leq \sum\limits_{\substack{i\in A: \\ \Delta_{i} > b}}\bigg\lbrace\dfrac{C_{1}(\rho_{s})T^{1-\rho_{s}}}{\Delta_{i}^{4\rho_{s}-1}} +\dfrac{64\rho_{s}\log{(\psi T\dfrac{\Delta_{i}^{4}}{16\rho_{s}^{2}})}}{\Delta_{i}} \\  
%& + 2\Delta_{i} \bigg\rbrace + \sum\limits_{\substack{i\in A\setminus A_{s^*}: \\ \Delta_{i} > b}} \dfrac{2C_{2}(\rho_{s})T^{1-\rho_{s}}}{\Delta_{i}^{4\rho_{s} -1}} \\ 
%& + \sum\limits_{\substack{i\in A\setminus A_{s^*}: \\ 0 <\Delta_{i}\leq b}}\dfrac{2C_{2}(\rho_{s})T^{1-\rho_{s}}}{b^{4\rho_{s} -1}} 
% + \max_{i\in A:\Delta_{i}\leq b}\Delta_{i}T,
%\end{align*}
% for all $b\geq \sqrt{\frac{K}{7T}}$, with $C_1$ and $C_2$ as defined in Theorem \ref{Result:Theorem:1}.
%%  (x) = \frac{2^{1+4x}x^{2x}}{\psi^{x}}$,  $C_2(x) = \frac{2^{2x+\frac{3}{2}}x^{2x}}{\psi^{x}}$, $\rho_{s}=\frac{1}{2} $ is the cluster elimination parameter, $\psi=K^{2}T$ is the exploration regulatory factor, $p$ is the number of clusters and $T$ is the horizon.
%\end{proposition}
%\begin{proof}
%See Appendix \ref{App:B}.
%\end{proof}
%

\section{Simulation experiments}
\label{sec:expts}
\input{expts}
%

\section{Conclusions and future work}
\label{sec:conclusions}
From a theoretical viewpoint, we conclude that the gap-dependent regret bound of EClusUCB is lower than MOSS and UCB-Improved. From the numerical experiments on settings with small gaps between optimal and sub-optimal mean rewards, we observed that EClusUCB outperforms several popular bandit algorithms,  including OCUCB. Also EClusUCB is remarkably stable for a large horizon and large number of arms and performs well across different types of distributions. While we exhibited better regret bounds for EClusUCB, it would be interesting future research to improve the theoretical analysis of EClusUCB to achieve the gap-independent regret bound of MOSS and OCUCB. This is also one of the first papers to apply clustering in stochastic MAB and another future direction is to use this in contextual or in distributed bandits. 

%Distributed bandits are a specific setup of MAB where a network of bandits collaborate with each other to identify the optimal arm(s) (see \cite{awerbuch2008competitive,liu2010distributed,szorenyi2013gossip,hillel2013distributed}). In our setting we can assign each of the $p$ clusters to individual bandits and at the end of each round they can share information synchronously to identify the optimal arm. This naturally results in a speedup of operation and helps in identifying the best arm faster. 


%\clearpage
%\newpage
% In the unusual situation where you want a paper to appear in the
% references without citing it in the main text, use \nocite
%\nocite{langley00}

\bibliography{biblio1}
%\bibliographystyle{apalike}
\bibliographystyle{plain}


%%%%%%%%%%%%%%%%%%%%%%%%%%%%%%%%%%%%%%%%%%%%%%%%%%%%%%%%%%%%
%%%%%%%%%%%%%%%%%%%%%%%%%%%%%%%%%%%%%%%%%%%%%%%%%%%%%%%%%%%%

\clearpage
\newpage
%\onecolumn	
\section*{Appendix}

%\newpage
\appendix
The Appendix is organized as follows. In Appendix~\ref{App:Table} we show the regret bound Table. In Appendix \ref{App:A} we prove Proposition \ref{proofTheorem:Prop:1}. In Appendix \ref{App:Proof:Corollary:1} we prove Corollary \ref{Result:Corollary:1} and in Appendix \ref{App:Proof:Corollary:2} we prove Corollary \ref{Result:Corollary:2}. Appendix \ref{App:E} deals with the idea of why we do clustering. The simple regret bound of EClusUCB and its associated Corollary is proved in \ref{App:SR_EClusUCB}. Algorithm \ref{alg:aclusucb}, Adaptive Clustered UCB is shown in Appendix \ref{App:AClusUCB}. More experiments are shown in Appendix \ref{App:MoreExp}.

\section{Regret Bound Table}
\label{App:Table}
\begin{table}[!h]
\caption{Gap-dependent regret bounds for different bandit algorithms}
\label{tab:regret-bds}
\begin{center}
\begin{tabular}{|c|c|}
\toprule
Algorithm  & Upper bound \\
\midrule
UCB1         &$O\left(\frac{K\log T}{\Delta}\right)$ \\\midrule
UCB-Improved &$O\left(\frac{K\log (T\Delta^{2})}{\Delta}\right)$ \\\midrule
MOSS	     &$O\left(\frac{K^{2}\log\left(T\Delta^{2}/K\right)}{\Delta}\right)$\\\midrule
EClusUCB      &$O\left(\frac{K\log\left(\frac{T\Delta^{2}}{\sqrt{\log (K)}}\right)}{\Delta}\right)$\\\bottomrule
\end{tabular}
\end{center}
\vspace*{-2em}
\end{table}

\section{Proof of Proposition 1}
\label{App:A}

\begin{proposition}
\label{proofTheorem:Prop:1}
The regret $R_T$ for EClusUCB-AE satisfies
\begin{align*}
&\E [R_{T}]\leq \sum\limits_{\substack{i\in A\\\Delta_{i} > b}}\bigg\lbrace\frac{C_{1}(\rho_{a})T^{1-\rho_{a}}}{\Delta_{i}^{4\rho_{a}-1}} + \Delta_{i}+\frac{32\rho_{a}\log{(\frac{\psi  T\Delta_{i}^{4}}{16\rho_{a}^{2}})}}{\Delta_{i}}
 +  \frac{C_{2}(\rho_{a})T^{1-\rho_{a}}}{\Delta_{i}^{4\rho_{a} -1}}  \bigg \rbrace \\
 %%%%%%%%%%%%%
 & +\sum\limits_{\substack{i\in A\\0 <\Delta_{i}\leq b}}\frac{C_{2}(\rho_{a})T^{1-\rho_{a}}}{b^{4\rho_{a} -1}}  + \max_{\substack{i\in A: \\ \Delta_{i}\leq b}}\Delta_{i}T,
\end{align*}
for all $b\geq\sqrt{\frac{K}{14 T}}$. In the above, $C_1, C_2$ are as defined in Theorem \ref{Result:Theorem:1}.
% (x) = \frac{2^{1+4x}x^{2x}}{\psi^{x}}$,  $C_2(x) = \frac{2^{2x+\frac{3}{2}}x^{2x}}{\psi^{x}}$, $\rho_{a}=\frac{1}{2}$ is the arm elimination parameter, $\psi=K^{2}T$ is the exploration regulatory factor, $p$ is the number of clusters and $T$ is the horizon.
\end{proposition}


\begin{proof}
Let $p=1$ such that all the arms in $A$ belongs to a single cluster. Hence, in EClusUCB-AE there is only arm elimination and no cluster elimination. Let, for each sub-optimal arm ${i}$, $m_{i}=\min{\lbrace m|\sqrt{\epsilon_{m}} < \dfrac{\Delta_{i}}{2} \rbrace}$. Also $\rho_{a}\in (0,1]$ is a constant in this proof. Let $A^{'}=\lbrace i\in A: \Delta_{i} > b \rbrace$ and $A^{''}=\lbrace i\in A: \Delta_{i} > 0 \rbrace$. Also $z_{i}$ denotes total number of times an arm $i$ has been pulled. In the $m$-th round, $n_{m}$ denotes the number of pulls allocated to the surviving arms in $B_{m}$. 

%The theoretical analysis remains same as we have always bounded the values of $\rho_{a}\in (0,1]$.

\subsection*{Case $a$: \textit{Some sub-optimal arm ${i}$ is not eliminated in round $m_{i}$ or before and the optimal arm ${*}\in B_{m_{i}}$}}
  
	Following the steps of Theorem \ref{Result:Theorem:1} Case $a1$, an arbitrary sub-optimal arm ${i}\in A^{'}$ can get eliminated only when the event,
	\begin{align}
	\hat{r}_{i}  \le r_{i} + c_{m_i} \text{ and } \label{eq:appA:armelim-casea}
 	\hat{r}^{*}\geq  r^{*} - c_{m_i}
	\end{align}
	
	takes place. So to bound the regret we need to bound the probability of the complementary event of these two conditions. Note that  $c_{m_{i}} = \sqrt{\frac{\rho_{a}\log (\psi T\epsilon_{m_{i}}^{2})}{2 n_{m_i}}}$. As arm elimination condition is being checked in every timestep, any arm $i$ cannot be pulled more than $z_i= n_{m_i}$ times or it will get eliminated. This is because in the $m_i$-th round $n_{m_{i}}=\dfrac{2\log{(\psi T\epsilon_{m_{i}}^{2})}}{\epsilon_{m_{i}}}$ and putting this in $c_{m_i}$ we get,
  $c_{m_i}=\sqrt{\dfrac{\rho_{a}\epsilon_{m_{i}}\log (\psi T\epsilon_{m_{i}}^{2})}{2*2 \log(\psi T\epsilon_{m_{i}}^{2})}}=\dfrac{\sqrt{\rho_{a}\epsilon_{m_{i}}}}{2} \leq  \sqrt{\rho_{a}\epsilon_{m_{i}+1}} < \dfrac{\Delta_{i}}{4} $, as $\rho_{a}\in (0,\frac{1}{2}]$.
%\leq\dfrac{\epsilon_{m}\sqrt{\ell_{m}}}{2\sqrt{w\ell_{m}}}\leq \dfrac{\epsilon_{m}}{2\sqrt{w}}$.
%  But in $\xi_{1}$, $\ell_{m}=2^{m}$.
%  Hence, $c\leq \dfrac{\epsilon_{m} 2^{m/2}}{4}$.
%  Also in the $m_i$-th round $n_{m_i}\leq n_{m_{*}}$ as $*\in B_{m}$ and so $c^{*}\leq c_{i}$. 
  Again, for ${i} \in A^{'}$, 
  \begin{align*}
\hat{r}_{i} + c_{m_i}&\leq r_{i} + 2c_{m_i} 
%&= \hat{r}_{i} + 4c_{m_{i}} - 2c_{m_{i}} \\
 < r_{i} + \Delta_{i} - 2c_{m_i}
 \leq r^{*} -2c_{m_i} 
 \leq \hat{r}^{*} - c_{m_i}
  \end{align*}

	Applying Chernoff-Hoeffding bound and considering independence of complementary of the two events in \ref{eq:appA:armelim-casea},
  \begin{align*}
\mathbb{P}\lbrace\hat{r}_{i}&\geq r_{i} + c_{m_i}\rbrace\leq \exp(-2c_{m_i}^{2}n_{m_{i}})
\leq \exp(-2 * \dfrac{\rho_{a}\log (\psi T\epsilon_{m_{i}}^{2})}{2 n_{m_{i}}} *n_{m_{i}})
\leq \dfrac{1}{(\psi T\epsilon_{m_{i}}^{2})^{\rho_{a}}}   
  \end{align*}
 
%$\leq \bigg(\dfrac{1}{4\psi T\epsilon_{m}^{2}}\bigg)^{D}$, as $\ell_{m}-1\leq D$
% \hspace*{2em}
 
Similarly, $\mathbb{P}\lbrace\hat{r}^{*}\leq r^{*} - c_{m_i}\rbrace\leq \dfrac{1}{(\psi  T\epsilon_{m_{i}}^{2})^{\rho_{a}}}$. Summing the two up, the probability that a sub-optimal arm ${i}$ is not eliminated on or before $m_{i}$-th round is  $\bigg(\dfrac{2}{(\psi T\epsilon_{m_{i}}^{2})^{\rho_{a}}}\bigg)$. 
 
Summing up over all arms in $A^{'}$ and bounding the regret for each arm $i\in A^{'}$ trivially by $T\Delta_{i}$, we obtain
   \begin{align*}
\sum_{i\in A^{'}}\bigg(\dfrac{2T\Delta_{i}}{(\psi T\epsilon_{m_{i}}^{2})^{\rho_{a}}}\bigg)
\leq\sum_{i\in A^{'}}\bigg(\dfrac{2T\Delta_{i}}{(\psi T\dfrac{\Delta_{i}^{4}}{16\rho_{a}^{2}})^{\rho_{a}}}\bigg)
&\leq \sum_{i\in A^{'}}\bigg(\dfrac{2^{1+4\rho_{a}}T^{1-\rho_{a}}\Delta_{i}}{\psi^{\rho_{a}}\Delta_{i}^{4\rho_{a}}}\bigg)
\leq \sum_{i\in A^{'}}\bigg(\dfrac{2^{1+4\rho_{a}}T^{1-\rho_{a}}}{\psi^{\rho_{a}}\Delta_{i}^{4\rho_{a}-1}}\bigg)\\  
%%%%%%%%%%%%%%%%%% 
& =\sum_{i\in A^{'}}\bigg(\dfrac{C_{1}(\rho_{a})T^{1-\rho_{a}}}{\Delta_{i}^{4\rho_{a}-1}}\bigg) \text{, where } C_1(x) = \frac{2^{1+4x}}{\psi^{x}}
   \end{align*}

%$C_1(x) = \frac{2^{1+4x}x^{2x}}{\psi^{x}}$ and $C_2(x) = \frac{2^{2x+\frac{3}{2}}x^{2x}}{\psi^{x}}$

% 
%Summing up over all arms in $A$ and bounding trivially by $T\Delta_{i}$,
%%\sum_{i\in A}\bigg(\dfrac{2}{T\epsilon_{m_{i}}^{2}}\bigg)\leq
% \hspace*{4em} $\sum_{i\in A}\bigg(\dfrac{2*4T\Delta_{i}}{T\epsilon_{m_{i}}\dfrac{\Delta_{i}}{2}^{2}}\bigg)\leq \sum_{i\in A}\bigg(\dfrac{8}{\epsilon_{m_{i}}\Delta_{i}}\bigg)\leq \sum_{i\in A}\bigg(\dfrac{32}{\Delta_{i}^{3}}\bigg)$


%\textbf{Case b1:} Either an arm $a_{i}$ is eliminated in round $m_{i}$ or before or else there is no optimal arm $a^{*}\in B_{m_{i}}$. 

\subsection*{Case $b$: \textit{Either an arm ${i}$ is eliminated in round $m_{i}$ or before or else there is no optimal arm ${*}\in B_{m_{i}}$ }}

\subsubsection*{Case $b1$: \textit{${*}\in B_{m_{i}}$ and each ${i}\in A^{'}$ is  eliminated on or before $m_{i}$ } }

 Since we are eliminating a sub-optimal arm ${i}$ on or before round $m_{i}$, it is pulled no longer than, 
 \begin{align*}
 z_{i} < \bigg\lceil\dfrac{2\log{(\psi T\epsilon_{m_{i}}^{2})}}{\epsilon_{m_{i}}}\bigg\rceil
 \end{align*}
%\hspace*{4em}
%%$, since $\sqrt{\rho_{a}\epsilon_{m_{i}}}\leq\dfrac{\Delta_{i}}{2}
So, the total contribution of ${i}$  till round $m_{i}$ is given by, 
\begin{align*}
&\Delta_{i}\bigg\lceil\dfrac{2\log{(\psi T\epsilon_{m_{i}}^{2})}}{\epsilon_{m_{i}}}\bigg\rceil
\leq\Delta_{i}\bigg\lceil\dfrac{2\log{(\psi T(\dfrac{\Delta_{i}}{2})^{4})}}{(\dfrac{\Delta_{i}}{2})^{2}}\bigg\rceil \text{, since } \sqrt{\epsilon_{m_{i}}} < \dfrac{\Delta_{i}}{2}\\
&\leq\Delta_{i}\bigg(1+\dfrac{32\log{(\psi T(\dfrac{\Delta_{i}}{2}})^{4})}{\Delta_{i}^{2}}\bigg)
\leq\Delta_{i}\bigg(1+\dfrac{32\log{(\psi T\dfrac{\Delta_{i}^{4}}{16})}}{\Delta_{i}^{2}}\bigg)
\end{align*} 
 
Summing over all arms in $A^{'}$ the total regret is given by, 
\begin{align*}
\sum_{i\in A^{'}}\Delta_{i}\bigg(1+\dfrac{32\log{(\psi T\dfrac{\Delta_{i}^{4}}{16}})}{\Delta_{i}^{2}}\bigg)
\end{align*}

\subsubsection*{Case $b2$: \textit{Optimal arm ${*}$ is eliminated by a sub-optimal arm  }}


	Firstly, if conditions of Case $a$ holds then the optimal arm ${*}$ will not be eliminated in round $m=m_{*}$ or it will lead to the contradiction that $r_{i}>r^{*}$. In any round $m_{*}$, if the optimal arm ${*}$ gets eliminated then for any round from $1$ to $m_{j}$ all arms ${j}$ such that $m_{j}< m_{*}$ were eliminated according to assumption in Case $a$. Let the arms surviving till $m_{*}$ round be denoted by $A^{'}$. This leaves any arm $a_{b}$ such that $m_{b}\geq m_{*}$ to still survive and eliminate arm ${*}$ in round $m_{*}$. Let such arms that survive ${*}$ belong to $A^{''}$. Also maximal regret per step after eliminating ${*}$ is the maximal $\Delta_{j}$ among the remaining arms ${j}$ with $m_{j}\geq m_{*}$.  Let $m_{b}=\min\lbrace m|\sqrt{\epsilon_{m}}<\dfrac{\Delta_{b}}{2}\rbrace$. Hence, the maximal regret after eliminating the arm ${*}$ is upper bounded by, 
\begin{align*}
&\sum_{m_{*}=0}^{max_{j\in A^{'}}m_{j}}\sum_{i\in A^{''}:m_{i}>m_{*}}\bigg(\dfrac{2}{(\psi  T\epsilon_{m_{*}}^{2})^{\rho_{a}}} \bigg).T\max_{j\in A^{''}:m_{j}\geq m_{*}}{\Delta}_{j}\\
%%%%%%%%%%%%%%%%%%%%%%%%%%%
&\leq\sum_{m_{*}=0}^{max_{j\in A^{'}}m_{j}}\sum_{i\in A^{''}:m_{i}>m_{*}}\bigg(\dfrac{2}{(\psi  T\epsilon_{m_{*}}^{2})^{\rho_{a}}} \bigg).T.2\sqrt{\epsilon_{m_{*}}}\\
%%%%%%%%%%%%%%%%%%%%%%%%%%
&\leq\sum_{m_{*}=0}^{max_{j\in A^{'}}m_{j}}\sum_{i\in A^{''}:m_{i}>m_{*}}4\bigg(\dfrac{T^{1-\rho_{a}}}{\psi^{\rho_{a}}\epsilon_{m_{*}}^{2\rho_{a}-\frac{1}{2}}} \bigg)\\
%%%%%%%%%%%%%%%%%%%%%%%%%%
&\leq\sum_{i\in A^{''}:m_{i}>m_{*}}\sum_{m_{*}=0}^{\min{\lbrace m_{i},m_{b}\rbrace}}\bigg(\dfrac{4T^{1-\rho_{a}}}{\psi^{\rho_{a}}2^{-(2\rho_{a}-\frac{1}{2})m_{*}}} \bigg)\\
%%%%%%%%%%%%%%%%%%%%%%%%%%
&\leq\sum_{i\in A^{'}}\bigg(\dfrac{4T^{1-\rho_{a}}}{\psi^{\rho_{a}}2^{-(2\rho_{a}-\frac{1}{2})m_{*}}} \bigg)+\sum_{i\in A^{''}\setminus A^{'}}\bigg(\dfrac{4T^{1-\rho_{a}}}{\psi^{\rho_{a}}2^{-(2\rho_{a}-\frac{1}{2})m_{b}}} \bigg)\\
%%%%%%%%%%%%%%%%%%%%%%%%%%
&\leq\sum_{i\in A^{'}}\bigg(\dfrac{4T^{1-\rho_{a}}*2^{2\rho_{a}-\frac{1}{2}}}{\psi^{\rho_{a}}\Delta_{i}^{4\rho_{a}-1}} \bigg)+\sum_{i\in A^{''}\setminus A^{'}}\bigg(\dfrac{4T^{1-\rho_{a}}*2^{2\rho_{a}-\frac{1}{2}}}{\psi^{\rho_{a}}b^{4\rho_{a}-1}} \bigg)\\
%%%%%%%%%%%%%%%%%%%%%%%%%%
&\leq\sum_{i\in A^{'}}\bigg(\dfrac{T^{1-\rho_{a}}2^{2\rho_{a}+\frac{3}{2}}}{\psi^{\rho_{a}}\Delta_{i}^{4\rho_{a}-1}} \bigg)+\sum_{i\in A^{''}\setminus A^{'}}\bigg(\dfrac{T^{1-\rho_{a}}2^{2\rho_{a}+\frac{3}{2}}}{\psi^{\rho_{a}}b^{4\rho_{a}-1}} \bigg)\\
%%%%%%%%%%%%%%%%%%%%%%%%%%
& = \sum_{i\in A^{'}}\bigg(\dfrac{ C_{2}(\rho_{a}) T^{1-\rho_{a}}}{\Delta_{i}^{4\rho_{a}-1}} \bigg)+\sum_{i\in A^{''}\setminus A^{'}}\bigg(\dfrac{C_{2(\rho_{a})}T^{1-\rho_{a}}}{b^{4\rho_{a}-1}} \bigg) \text{, where } C_2(x) = \frac{2^{2x+\frac{3}{2}}}{\psi^{x}}
\end{align*}

%\text{, since } \sqrt{\rho_{a}\epsilon_{m}}<\dfrac{\Delta_{i}}{2}

 
Summing up \textbf{Case a} and \textbf{Case b}, the total regret till round $m$ is given by,
\begin{align*}
 R_{T} \leq &\sum\limits_{i\in A:\Delta_{i} > b}\bigg\lbrace\bigg(\dfrac{C_{1}(\rho_{a})T^{1-\rho_{a}}}{\Delta_{i}^{4\rho_{a}-1}}\bigg) + \bigg(\Delta_{i}+\dfrac{32\log{(\psi  T\dfrac{\Delta_{i}^{4}}{16})}}{\Delta_{i}}\bigg)
  +  \bigg(\dfrac{C_{2}(\rho_{a})T^{1-\rho_{a}}}{\Delta_{i}^{4\rho_{a} -1}} \bigg) \bigg \rbrace\\
  & +\sum\limits_{i\in A:0 < \Delta_{i}\leq b}\bigg(\dfrac{C_{2}(\rho_{a})T^{1-\rho_{a}}}{\psi^{\rho_{a}}b^{4\rho_{a} -1}} \bigg) + \max_{i\in A:\Delta_{i}\leq b}\Delta_{i}T
\end{align*}

% R_{T} \leq &\sum\limits_{i\in A:\Delta_{i}\geq b}\bigg\lbrace\bigg(\dfrac{2^{1+4\rho_{a}}\rho_{a}^{2\rho_{a}}T^{1-\rho_{a}}}{\psi^{\rho_{a}}\Delta_{i}^{4\rho_{a}-1}}\bigg) + \bigg(\Delta_{i}+\dfrac{32\rho_{a}\log{(\psi  T\dfrac{\Delta_{i}^{4}}{16\rho_{a}^{2}})}}{\Delta_{i}}\bigg)\\
%&  +  \bigg(\dfrac{T^{1-\rho_{a}}\rho_{a}^{2\rho_{a}}2^{2\rho_{a}+\frac{3}{2}}}{\psi^{\rho_{a}}\Delta_{i}^{4\rho_{a} -1}} \bigg) \bigg \rbrace+\sum\limits_{i\in A:0\leq\Delta_{i}\leq b}\bigg(\dfrac{T^{1-\rho_{a}}\rho_{a}^{2\rho_{a}}2^{2\rho_{a}+\frac{3}{2}}}{\psi^{\rho_{a}}b^{4\rho_{a} -1}} \bigg) + max_{i:\Delta_{i}\leq b}\Delta_{i}T
  
\end{proof}

%%%%%%%%%%%%%%%%%%%%%%%%%%%%%%%%%%%%%%%%%%%%%%%%%
%%Corollary 3 not included
%%%%%%%%%%%%%%%%%%%%%%%%%%%%%%%%%%%%%%%%%%%%%%%%%
%\begin{corollary}
%\label{App:Proof:Corollary:3}
%For $\rho_{a}=1$ in the result of proposition \ref{proofTheorem:Prop:1} for ClusUCB-AE, we get a regret bound of 
%
% \begin{align*}
% &\sum\limits_{i\in A:\Delta_{i} > b}\bigg(\Delta_{i} + \dfrac{44}{\psi(\Delta_{i})^{3}} + \dfrac{32\log{(\psi T\Delta_{i}^{4})}}{\Delta_{i}}\bigg) + \sum\limits_{i\in A:0< \Delta_{i}\leq b}\dfrac{12}{\psi b^{3}}
% \end{align*}.
%\end{corollary}
%
%%\begin{proof}
%%The proof of this corollary is given in Appendix \ref{App:Proof:Corollary:3}.
%%\end{proof}
%
%\begin{proof}
%In the result of Proposition $1$ if we take $\rho_{a}=1$ then the regret bound becomes $ \sum\limits_{i\in A:\Delta_{i} > b}\bigg(\Delta_{i} + \dfrac{44}{\psi(\Delta_{i})^{3}} + \dfrac{32\log{(\psi T\Delta_{i}^{4})}}{\Delta_{i}}\bigg) + \sum\limits_{i\in A:0< \Delta_{i}\leq b}\dfrac{12}{\psi b^{3}}$. From the result we can see that for small $\Delta_{i}$ and large $K$, the terms like $ \sum\limits_{i\in A:\Delta_{i} > b}\bigg(\dfrac{44}{\psi(\Delta_{i})^{3}}\bigg) + \sum\limits_{i\in A:0 < \Delta_{i}\leq b}\dfrac{12}{\psi b^{3}}$ can become the dominant term in the regret rather than $\sum\limits_{i\in A:\Delta_{i} > b}\dfrac{32\log{(\psi T\Delta_{i}^{4})}}{\Delta_{i}}$. Intuitively, this actually suggests that the algorithm is trying to eliminate arms with too low exploration and so the probability of elimination is low and error(risk) is high. For this essentially we introduce $\rho_{a},\rho_{s}$ and $\psi$ and by carefully defining their values enables us to eliminate arms and clusters aggressively and thereby reduce those two terms. 
%\end{proof}

%%%%%%%%%%%%%%%%%%%%%%%%%%
%The proof for ClusUCB-CE not included
%%%%%%%%%%%%%%%%%%%%%%%%%%


%\section{Proof of Proposition 2}
%\label{App:B}
%
%\begin{figure}
%\includegraphics[scale=0.3]{img/diagCluster.jpg}
%\caption{Cluster Elimination}
%\label{Fig:ClusFig}
%\end{figure}
%
%An illustrative diagram explaining Cluster Elimination is given in \textbf{Figure \ref{Fig:ClusFig}}. A slight modification to the algorithm allows us to do cluster elimination without any arm elimination. By taking $p>1$, removing the arm elimination condition, stopping when we are just left with one cluster and pulling the $max\lbrace \hat{r}_{i}\rbrace$, where ${i}\in B_{m}$ we can achieve this. We also take $\rho_{s}\in (0,1]$ as a constant in this proof whereby in Corollary \ref{Result:Corollary:1} and \ref{Result:Corollary:2} we use the different definitions. The theoretical analysis remains same as we have always bounded the values of $\rho_{s}\in (0,1]$.
% 
%
%\begin{proof}
%Let $C_{g_{s_{k}}}=\lbrace \hat{r}_{a_{\max_{s_{k}}}} | \forall s_{k}\in S \rbrace$, that is let $C_{g_{s_{k}}}$ be the set of all arms which has the maximum estimated payoff arms from their respective clusters in the $g_{s_{k}}$-th round.  
%Let, for each sub-optimal cluster arm $a_{\max_{s_{k}}}\in C_{g_{s_{k}}}$, $g_{s_{k}}=\min{\lbrace g|\sqrt{\rho_{s}\epsilon_{g}} < \dfrac{\Delta_{a_{\max_{s_{k}}}}}{2} \rbrace}$. So, $g_{s_{k}}$ be the first round when $\sqrt{\rho_{s}\epsilon_{g_{s_{k}}}} < \dfrac{\Delta_{a_{\max_{s_{k}}}}}{2}$ where $a_{\max_{s_{k}}}\in C_{g_{s_{k}}}$ is the maximum payoff arm in cluster $s_{k}$. Here, $a_{\max_{s_{k}}}$ is called cluster arm. Here, $A_{{s_{k}}}$ denotes the arm set in the cluster $s_{k}$. Let $A_{s_{k}}^{'}=\lbrace i\in A_{s_{k}}: \Delta_{i}> b\rbrace$, $A_{s_{k}}^{''}=\lbrace i\in A_{s_{k}}: \Delta_{i} > 0\rbrace$, $A^{'}=\lbrace i\in A: \Delta_{i}> b\rbrace$ and $A^{''}=\lbrace i\in A: \Delta_{i} > 0\rbrace$.
%
%%The parameter $\rho_{s}$ is introduced just to make sure that the cluster elimination is a more aggressive elimination than arm elimination.
%\subsection*{Case a: \textit{ Some sub-optimal cluster arm $a_{max_{s_{k}}}$ is not eliminated in round $g_{s_{k}}$ or before with $* \in C_{g_{s_{k}}}$ }}
%  
%	Following the steps of Theorem \ref{Result:Theorem:1} Case $a2$, an arbitrary sub-optimal arm ${i}\in A^{'}$ can get eliminated only when the event,
%	\begin{align}
%	\hat{r}_{a_{\max_{s_{k}}}}  \le r_{a_{\max_{s_{k}}}} + c_{m_{i}} \text{ and } \label{eq:appB:armelim-casea}
% 	\hat{r}^{*}\geq  r^{*} - c_{m_{i}}
%	\end{align}
%	
%	takes place. So to bound the regret we need to bound the probability of the complementary event of these two conditions.  
%  
%  
%  Putting the value of $n_{g_{s_{k}}}=\dfrac{2\log{(\psi T\epsilon_{g_{s_{k}}}^{2})}}{\epsilon_{g_{s_{k}}}}$ in $c_{g_{s_{k}}}$ we get,
%  \begin{align*}
%  c_{g_{s_{k}}}= & \sqrt{\dfrac{\rho_{s}*\epsilon_{g_{s_{k}}}\log (\psi  T\epsilon_{g_{s_{k}}}^{2})}{2*2 \log(\psi T\epsilon_{g_{s_{k}}}^{2})}}\\
%  &=\sqrt{\dfrac{\rho_{s}\epsilon_{g_{s_{k}}}}{2}}\\
%  &=\sqrt{\rho_{s}\epsilon_{g_{s_{k}}+1}} < \dfrac{\sqrt{\rho_{s}}\Delta_{a_{\max_{s_{k}}}}}{4} < \dfrac{\Delta_{a_{\max_{s_{k}}}}}{4}
%  \end{align*}
%%  $c_{g_{s_{k}}}=\sqrt{\dfrac{\rho_{s}*\epsilon_{g_{s_{k}}}\log (\psi T\epsilon_{g_{s_{k}}}^{2})}{2*2 \log(\psi T\epsilon_{g_{s_{k}}}^{2})}}=\sqrt{\dfrac{\rho_{s}\epsilon_{g_{s_{k}}}}{2}} = \sqrt{\rho_{s}\epsilon_{g_{s_{k}}+1}} < \dfrac{\sqrt{\rho_{s}}\Delta_{a_{max_{s_{k}}}}}{4} < \dfrac{\Delta_{a_{max_{s_{k}}}}}{4} $
%
%  Again, for $a_{\max_{s_{k}}}, * \in C_{g_{s_{k}}}$, 
%  \begin{align*}
%  \hat{r}_{a_{\max_{s_{k}}}} + c_{g_{s_{k}}}\leq r_{a_{\max_{s_{k}}}} + 2c_{g_{s_{k}}} &= \hat{r}_{a_{\max_{s_{k}}}} + 4c_{g_{k}} - 2c_{g_{s_{k}}}\\
%  &< r_{a_{\max_{s_{k}}}} + \Delta_{a_{\max_{s_{k}}}} - 2c_{g_{s_{k}}}\\
%  &= r^{*} -2c_{g_{s_{k}}}\\
%  &\leq \hat{r}^{*} - c_{g_{s_{k}}}
%  \end{align*}
%   
%
%	Applying Chernoff-Hoeffding bound and considering independence of complementary of the two events in \ref{eq:appB:armelim-casea},
% 
% \begin{align*}
% \mathbb{P}\bigg\lbrace\hat{r}^{*} \leq r^{*} - c_{g_{s_{k}}}\bigg\rbrace&\leq exp(-2c_{g_{s_{k}}}^{2}n_{g_{s_{k}}})\\
% &\leq exp(-2 * \dfrac{\rho_{s}\log ( \psi T\epsilon_{g_{s_{k}}}^{2})}{2 n_{g_{s_{k}}}} *n_{g_{s_{k}}})\\
% &\leq \dfrac{1}{(\psi T\epsilon_{g_{k}}^{2})^{\rho_{s}}}
% \end{align*}
%
% 
%Similarly, $\mathbb{P}\bigg\lbrace\hat{r}_{a_{max_{s_{k}}}}\geq r_{a_{max_{s_{k}}}} + c_{g_{s_{k}}}\bigg\rbrace\leq \dfrac{1}{(\psi T\epsilon_{g_{s_{k}}}^{2})^{\rho_{s}}}$
% 
%Summing, the two up, the probability that a sub-optimal cluster arm $a_{max_{s_{k}}}\in C_{g_{s_{k}}}$ is not eliminated in $g_{s_{k}}$-th round is  $\bigg(\dfrac{2}{(\psi  T\epsilon_{g_{s_{k}}}^{2})^{\rho_{s}}}\bigg)$. 
%  Now, for each round $g_{s_{k}}$, all the elements of $C_{g_{s_{k}}}$ are the respective max payoff arms of their cluster $s_{k}$, that is all the other arms in their respective clusters have performed worse than them. Hence, since $A_{s_{k}}^{'}\supset C_{g_{s_{k}}}$, we are pulling all the surviving arms equally in each round and since clusters are fixed so we can bound the maximum probability that a sub-optimal arm ${j}\in A^{'}_{s_{k}}$  and ${j}\in s_{k}$ such that $a_{max_{s_{k}}}\in C_{g_{s_{k}}}$ is not eliminated on or before the $g_{s_{k}}$-th round by the same probability of 
%  
%  %, \forall s_{k}\in S_{g_{s_{k}}}
%\begin{align*}
%\bigg(\dfrac{2}{(\psi T\epsilon_{g_{s_{k}}}^{2})^{\rho_{s}}}\bigg)
%\end{align*}  
% 
% 
%Summing up over all arms in $s_{k}$ and bounding the regret trivially by $T\Delta_{i}$,
%\begin{align*}
%\sum_{i\in A_{s_{k}}^{'}}\bigg(\dfrac{2T\Delta_{i}}{(\psi T\epsilon_{g_{s_{k}}}^{2})^{\rho_{s}}}\bigg)
%\end{align*}
%
% 
%Summing up over all $p$ clusters and bounding the regret for each arm $i\in A_{s_{k}}^{'} $ trivially by $T\Delta_{i}$,
% \begin{align*}
% \sum_{k=1}^{p}\sum_{i\in A_{s_{k}}^{'}}\bigg(\dfrac{2T\Delta_{i}}{(\psi T\dfrac{\Delta_{i}^{4}}{16\rho_{s}^{2}})^{\rho_{s}}}\bigg) &= \sum_{i\in A^{'}}\bigg(\dfrac{2T\Delta_{i}}{(\psi  T\dfrac{\Delta_{i}^{4}}{16\rho_{s}^{2}})^{\rho_{s}}}\bigg) \\
% &\leq \sum_{i\in A^{'}}\bigg(\dfrac{2^{1+4\rho_{s}}T^{1-\rho_{s}}\rho_{s}^{2\rho_{s}}\Delta_{i}}{\psi^{\rho_{s}}\Delta_{i}^{4\rho_{s}}}\bigg)\\
% &\leq \sum_{i\in A^{'}}\bigg(\dfrac{2^{1+4\rho_{s}}\rho_{s}^{2\rho_{s}}T^{1-\rho_{s}}}{\psi^{\rho_{s}}\Delta_{i}^{4\rho_{s}-1}}\bigg)\\
% &= \sum_{i\in A^{'}}\bigg(\dfrac{C_{1}(\rho_{s})T^{1-\rho_{s}}}{\Delta_{i}^{4\rho_{s}-1}}\bigg) \text{, where } C_1(x) = \frac{2^{1+4x}x^{2x}}{\psi^{x}}
% \end{align*}
% 
%
%
%\subsection*{Case b: \textit{For each arm $i$, either ${i}$ is eliminated in round $g_{s_{k}}$ or before or there is no optimal arm ${*}$ in $C_{g_{s_{k}}}$ }}
%
%\subsubsection*{Case b1: \textit{${*}\in C_{g_{s_{k}}}$ for each arm $i \in A'$ and cluster $s_{k}$ eliminated on or before $g_{s_{k}}$} }
%	
%	Again, in the $g_{s_{k}}$-th round, the maximum total elements in the cluster $s_{k}$ can be no more than $\ell=\bigg\lceil \dfrac{K}{p}\bigg\rceil$.
% 
%Also, since we are eliminating a sub-optimal cluster arm $a_{\max_{s_{k}}}\in C_{g_{s_{k}}}$ on or before round $g_{s_{k}}$, it is pulled (along with all the other arms in that cluster) no longer than,
% \begin{align*}
% &n_{g_{s_{k}}}=\bigg\lceil\dfrac{2\log{(\psi T\epsilon_{g_{s_{k}}}^{2})}}{\epsilon_{g_{s_{k}}}}\bigg\rceil
% \end{align*}
%
%So, the total contribution of $a_{\max_{s_{k}}}$  along with all the other arms in the cluster till round $g_{s_{k}}$ is given by,
% \begin{align*}
% &\sum_{i\in A_{s_{k}}}\Delta_{i}\bigg\lceil\dfrac{2\log{(\psi T\epsilon_{g_{s_{k}}}^{2})}}{\epsilon_{g_{s_{k}}}}\bigg\rceil\\
% &\leq\sum_{i\in A_{s_{k}}^{'}}\Delta_{i}\bigg\lceil\dfrac{2\log{(\psi T(\dfrac{\Delta_{i}}{2\sqrt{\rho_{s}}})^{4})}}{(\dfrac{\Delta_{i}}{2\sqrt{\rho_{s}}})^{2}}\bigg\rceil \text{, since }\sqrt{\rho_{s}\epsilon_{g_{s_{k}}}} <\dfrac{\Delta_{a_{max_{s_{k}}}}}{2} <  \dfrac{\Delta_{i}}{2} \text{, as } {r}_{a_{max_{s_{k}}}}>{r}_{i},\forall i\in s_{k}\\
% &\leq\sum_{i\in A_{s_{k}}^{'}}\Delta_{i}\bigg(1+\dfrac{32*\rho_{s}*\log{(\psi T(\dfrac{\Delta_{i}}{2\sqrt{\rho_{s}}})^{4})}}{\Delta_{i}^{2}}\bigg)\\
% &\leq\sum_{i\in A_{s_{k}}^{'}}\Delta_{i}\bigg(1+\dfrac{32\rho_{s}\log{(\psi T\dfrac{\Delta_{i}^{4}}{16\rho_{s}^{2}})}}{\Delta_{i}^{2}}\bigg)
% \end{align*}
%
% 
%Summing over all $p$ clusters the total regret is given by,
% 
%\begin{align*}
%&\sum_{k=1}^{p}\sum_{i\in A_{s_{k}}^{'}}\Delta_{i}\bigg(1+\dfrac{32\rho_{s}\log{(\psi  T\dfrac{\Delta_{i}^{4}}{16\rho_{s}^{2}})}}{\Delta_{i}^{2}}\bigg)\\
%&\leq\sum_{i\in A^{'}}\Delta_{i}\bigg(1+\dfrac{32\rho_{s}\log{(\psi T\dfrac{\Delta_{i}^{4}}{16\rho_{s}^{2}}})}{\Delta_{i}^{2}}\bigg)
%\end{align*}
%
%
%\subsubsection*{Case b2: \textit{$s^{*}$ is eliminated by some sub-optimal cluster.} } 
%	
%	Let $C_{g}^{'}=\lbrace a_{max_{s_{k}}}\in A^{'}|\forall s_{k}\in S \rbrace$ and $C_{g}^{''}=\lbrace a_{max_{s_{k}}}\in A^{''}|\forall s_{k}\in S \rbrace$. Firstly, if conditions of case $b1$ holds then the optimal arm ${*}\in C_{g_{s_{k}}}$ will not be eliminated in round $g_{s_{k}}=g_{*}$ or it will lead to the contradiction that $r_{a_{\max_{s_{k}}}}>r^{*}$ where $a_{\max_{s_{k}}},{*}\in C_{g_{s_{k}}}$. In any round $g_{*}$, if the optimal arm ${*}$ gets eliminated then for any round from $1$ to $g_{s_{j}}$ all arms $a_{\max_{s_{j}}}\in C_{g_{s_{k}}},\forall s_{j}\neq s^{*}$ such that $g_{s_{j}}< g_{*}$ were eliminated according to assumption in Case $a$. Let, the arms surviving till $g_{*}$ round be denoted by $C_{g}^{'}$. This leaves any arm $a_{s_{b}}$ such that $g_{s_{b}}\geq g_{*}$ to still survive and eliminate arm ${*}$ in round $g_{*}$. Let, such arms that survive ${*}$ belong to $C_{g}^{''}$. Also maximal regret per step after eliminating ${*}$ is the maximal $\Delta_{a_{\max_{s_{j}}}}$ among the remaining arms ${a_{\max_{s_{j}}}}\in C_{g_{s_{j}}}$ with $g_{s_{j}}\geq g_{*}$. Hence, the maximal regret after eliminating the arm ${*}$ is upper bounded by, 
% \begin{align*}
% &\sum_{g_{*}=0}^{max_{a_{\max_{s_{j}}}\in C_{g}^{'}}g_{s_{j}}}\sum_{\substack{a_{\max_{s_{k}}}\in C_{g}^{''}: \\ g_{s_{k}} \geq g_{*}}}\bigg(\dfrac{2}{(\psi T\epsilon_{g_{s^{*}}}^{2})^{\rho_{s}}} \bigg).T\max_{\substack{a_{\max_{s_{j}}}\in C_{g}^{''}: \\ g_{s_{j}}\geq g_{*}}}{\Delta}_{a_{\max_{s_{j}}}}
% \end{align*}
%%Let $g_{s_{b}}=\min\lbrace g|\sqrt{\rho_{s}\epsilon_{g}}<\dfrac{\Delta_{a_{\max_{s_{b}}}}}{2}\rbrace$.
%But, we know that for any round $g$, elements of $C_{g}$ are the best performers in their respective clusters. So, taking that into account and $A'\supset C_{g}^{'}$ and $A''\supset C_{g}^{''}$ the regret can be bounded by,
%%\text{, since }\sqrt{\rho_{s}\epsilon_{g_{s_{j}}}} < \dfrac{\Delta_{j}}{2} <  \dfrac{\Delta_{j}}{2} \text{, as }{r}_{a_{s_{j}}}>{r}_{j},\forall j\in s_{j}
%\begin{align*}
% & \sum_{g_{*}=0}^{max_{j\in A^{'}\setminus A_{s^*}^{'}}g_{s_{j}}}\sum_{i\in A^{''}\setminus A_{s^*}^{'}:g_{s_{k}}>g_{*}}\bigg(\dfrac{2}{(\psi T\epsilon_{g_{s_{k}}}^{2})^{\rho_{s}}} \bigg).T\max_{j\in A^{''}:g_{s_{j}}\geq g_{*}}{\Delta}_{j}\\
% &\leq\sum_{g_{*}=0}^{max_{j\in A^{'}\setminus A_{s^*}^{'}}g_{s_{j}}}\sum_{i\in A^{''}\setminus A_{s^*}^{'}:g_{s_{k}}>g_{*}}\bigg(\dfrac{2}{(\psi T\epsilon_{g_{s_{k}}}^{2})^{\rho_{s}}} \bigg).T.2\sqrt{\rho_{s}\epsilon_{g_{s_{j}}}}\\
% &\leq\sum_{g_{*}=0}^{max_{j\in A^{'}\setminus A_{s^*}^{'}}g_{s_{j}}}\sum_{i\in A^{''}\setminus A_{s^*}^{'}:g_{s_{k}}>g_{*}}\bigg(\dfrac{4T^{1-\rho_{s}}}{\psi^{\rho_{s}}\epsilon_{g_{s_{k}}}^{2\rho_{s} - \frac{1}{2}}} \bigg)\\
% &\leq\sum_{i\in A^{''}\setminus A_{s^*}^{'}:g_{s_{k}}>g_{*}}\sum_{g_{*}=0}^{\min{\lbrace g_{s_{k}},g_{s_{b}}\rbrace}}\bigg(\dfrac{4T^{1-\rho_{s}}}{\psi^{\rho_{s}}2^{({2\rho_{s} - \frac{1}{2}})g_{*}}} \bigg) \\
% &\leq\sum_{i\in A^{'}\setminus A_{s^*}^{'}}\bigg(\dfrac{4T^{1-\rho_{s}}}{\psi^{\rho_{s}}2^{({2\rho_{s} - \frac{1}{2}})g_{*}}} \bigg)+\sum_{i\in A^{''}\setminus A^{'}\cup A_{s^*}^{'}}\bigg(\dfrac{4T^{1-\rho_{s}}}{\psi^{\rho_{s}}2^{({2\rho_{s} - \frac{1}{2}})g_{s_{b}}}} \bigg)\\ 
% &\leq\sum_{i\in A^{'}\setminus A_{s^*}^{'}}\bigg(\dfrac{4\rho_{s}^{2\rho_{s}}T^{1-\rho_{s}}*2^{2\rho_{s}-\frac{1}{2}}}{\psi^{\rho_{s}}\Delta_{i}^{4\rho_{s}-1}} \bigg)+\sum_{i\in A^{''}\setminus A^{'}\cup A_{s^*}^{'}}\bigg(\dfrac{4\rho_{s}^{2\rho_{s}}T^{1-\rho_{s}}*2^{2\rho_{s}-\frac{1}{2}}}{\psi^{\rho_{s}}b^{4\rho_{s}-1}} \bigg)\\
% &\leq\sum_{i\in A^{'}\setminus A_{s^*}^{'}}\bigg(\dfrac{T^{1-\rho_{s}}\rho_{s}^{2\rho_{s}}2^{2\rho_{s}+\frac{3}{2}}}{\psi^{\rho_{s}}\Delta_{i}^{4\rho_{s}-1}} \bigg)+\sum_{i\in A^{''}\setminus A^{'}\cup A_{s^*}^{'}}\bigg(\dfrac{T^{1-\rho_{s}}\rho_{s}^{2\rho_{s}}2^{2\rho_{s}+\frac{3}{2}}}{\psi^{\rho_{s}}b^{4\rho_{s}-1}} \bigg)\\
% & = \sum_{i\in A^{'}\setminus A_{s^*}^{'}}\bigg(\dfrac{C_{2}(\rho_{s})T^{1-\rho_{s}}}{\Delta_{i}^{4\rho_{s}-1}} \bigg)+\sum_{i\in A^{''}\setminus A^{'}\cup A_{s^*}^{'}}\bigg(\dfrac{C_{2}(\rho_{s})T^{1-\rho_{s}}}{b^{4\rho_{s}-1}} \bigg) \text{, where } C_2(x) = \frac{2^{2x+\frac{3}{2}}x^{2x}}{\psi^{x}}
%\end{align*}
%
%\subsubsection*{Case b3: \textit{${*}$ is not in $C_{g_{s_{k}}}$, but belongs to $B_{g_{s_{k}}}$} } 
%
%In this case the optimal arm ${*}\in s^{*}$ is not eliminated, also $s^{*}$ is not eliminated. So, for all sub-optimal arms $i$ in $A'$ which gets eliminated on or before $g_{s_{k}}$ will get pulled no less than $\bigg\lceil\dfrac{2\log{(\psi T\epsilon_{g_{s_{k}}}^{2})}}{\epsilon_{g_{s_{k}}}}\bigg\rceil$ number of times, which leads to the following bound the contribution to the expected regret, as in Case $b1$:
%
%\begin{align*}
% &\sum_{i\in A^{'}}\bigg\lbrace \Delta_{i}+\dfrac{32\rho_{s}\log{(\psi T\dfrac{\Delta_{i}^{4}}{16\rho_{s}^{2}})}}{\Delta_{i}} \bigg\rbrace 
%\end{align*} 
%
%For arms $a_i \notin s^*$, the contribution to the regret cannot be greater than that in Case $b2$. So the regret is bounded by,
%
%\begin{align*}
%\sum_{i\in A^{'}\setminus A_{s^*}^{'}}\dfrac{C_{2}(\rho_{s})T^{1-\rho_{s}}}{\Delta_{i}^{4\rho_{s}-1}} +\sum_{i\in A^{''}\setminus A^{'} \cup A_{s^*}^{'}}\dfrac{C_{2}(\rho_{s})T^{1-\rho_{s}}}{b^{4\rho_{s}-1}}
%\end{align*}
%
%
% 
%Summing up \textbf{Case a} and \textbf{Case b}, the total regret till round $g$ is given by,
%\begin{align*}
%  R_{T} \leq &  \sum_{i\in A:\Delta_{i} > b}\bigg\lbrace\underbrace{\bigg(\dfrac{C_{1}(\rho_{s})T^{1-\rho_{s}}}{\Delta_{i}^{4\rho_{s}-1}}\bigg)}_{\text{case a}} + \underbrace{\bigg(2\Delta_{i}+\dfrac{64\rho_{s}\log{(\psi T\dfrac{\Delta_{i}^{4}}{16\rho_{s}^{2}})}}{\Delta_{i}} }_{\text{case b1+b3}} \bigg\rbrace
%  \\&+ \underbrace{\sum_{i\in A\setminus A_{s^*}: \Delta_{i} > b}\bigg(\dfrac{2C_{2}(\rho_{s})T^{1-\rho_{s}}}{\Delta_{i}^{4\rho_{s} -1}} \bigg)}_{\text{case b2}} \bigg\rbrace  + \underbrace{\sum_{i\in A\setminus A_{s^*}:0 < \Delta_{i}\leq b}\bigg(\dfrac{2C_{2}(\rho_{s})T^{1-\rho_{s}}}{\Delta_{i}^{4\rho_{s} -1}} \bigg)}_{\text{case b2}}  + max_{i\in A:\Delta_{i}\leq b}\Delta_{i}T
%\end{align*}
%\end{proof}


\section{Proof of Corollary 1}
\label{App:Proof:Corollary:1}
\begin{proof}
Here we take $\psi=\dfrac{T}{196 \log(K)}$, $\rho_{a}=\dfrac{1}{2}$ and $\rho_{s}=\dfrac{1}{2}$. Taking into account Theorem \ref{Result:Theorem:1} below for all $b\geq \sqrt{\dfrac{K}{14 T}}$

\begin{align*}
&\E [R_{T}]\leq 
\sum\limits_{\substack{i\in A_{s^{*}},\\\Delta_{i} > b}}\bigg\lbrace \dfrac{C_1(\rho_{a})T^{1-\rho_{a}}}{\Delta_{i}^{4\rho_{a}-1}} + \Delta_{i}
 + \frac{32\log{(\psi T\dfrac{\Delta_{i}^{4}}{16})}}{\Delta_{i}} \bigg\rbrace
 + \! \! \sum\limits_{\substack{i\in A,\\\Delta_{i} > b}} \bigg\lbrace 2\Delta_{i}+
\dfrac{C_1(\rho_{s})T^{1-\rho_{s}}}{\Delta_{i}^{4\rho_{s}-1}} \\
&+ \frac{32\log{(\psi T\dfrac{\Delta_{i}^{4}}{16})}}{\Delta_{i}} 
+ \dfrac{32\log{(\psi T\dfrac{\Delta_{i}^{4}}{16})}}{\Delta_{i}}\bigg\rbrace 
%%%
+ \sum\limits_{\substack{i\in A_{s^{*}},\\ \Delta_{i} > b}} 
\frac{C_2(\rho_{a})T^{1-\rho_{a}}}{\Delta_{i}^{4\rho_{a}-1}}
+\sum\limits_{\substack{i\in A_{s^{*}},\\0 < \Delta_{i}\leq b}}\frac{C_2(\rho_a)T^{1-\rho_{a}}}{b^{4\rho_{a} -1}}\\ 
 & + \sum_{i\in A\setminus A_{s^*}: \Delta_{i} > b}\dfrac{2C_{2}(\rho_{s})T^{1-\rho_{s}}}{\Delta_{i}^{4\rho_{s}-1}} +\sum_{i\in A\setminus A_{s^*}: 0 < \Delta_{i} \leq b}\dfrac{2C_{2}(\rho_{s})T^{1-\rho_{s}}}{b^{4\rho_{s}-1}} +
 \!+\! \max\limits_{i:\Delta_{i}\leq b}\Delta_{i}T
\end{align*}

and putting the parameter values in the above Theorem \ref{Result:Theorem:1} result,
	\begin{align*}
	\sum_{i\in A_{s^{*}}:\Delta_{i} > b}\bigg(\dfrac{T^{1-\rho_{a}}2^{1+4\rho_{a}}}{\psi^{\rho_{a}}\Delta_{i}^{4\rho_{a}-1}} \bigg)= \sum_{i\in A_{s^{*}}:\Delta_{i} > b}\bigg(\dfrac{T^{1-\frac{1}{2}}2^{1+4*\frac{1}{2}}}{(\frac{T}{196 \log (K)})^{\frac{1}{2}}\Delta_{i}^{4*\frac{1}{2}-1}} \bigg)=\sum_{i\in A_{s^{*}}:\Delta_{i} > b}\dfrac{112\sqrt{\log (K)}}{\Delta_{i}}
	\end{align*}
	
	Similarly for the term,
	\begin{align*}
	\sum_{i\in A:\Delta_{i} > b}\bigg(\dfrac{T^{1-\rho_{s}}2^{1+4\rho_{s}}}{\psi^{\rho_{s}}\Delta_{i}^{4\rho_{s}-1}} \bigg) = \sum_{i\in A:\Delta_{i} > b}\dfrac{112\sqrt{\log (K)}}{\Delta_{i}}
	\end{align*}		
			
	
	For the term involving arm pulls,
	\begin{align*}
	\sum_{i\in A:\Delta_{i} > b}\dfrac{32\log{(\psi T\dfrac{\Delta_{i}^{4}}{16})}}{\Delta_{i}} 
	\leq  \sum_{i\in A:\Delta_{i} > b}\dfrac{32\log{(T^{2}\dfrac{\Delta_{i}^{4}}{16\log (K)})}}{\Delta_{i}}\approx \sum_{i\in A:\Delta_{i} > b}\dfrac{64\log{(T\dfrac{\Delta_{i}^{2}}{\sqrt{\log (K)}})}}{\Delta_{i}}
	\end{align*}		
	 Similarly the term, 
	
	\begin{align*}
	\sum_{i\in A:\Delta_{i} > b}\dfrac{32\log{(\psi T\dfrac{\Delta_{i}^{4}}{16\rho_{a}^{2}})}}{\Delta_{i}}\approx \sum_{i\in A:\Delta_{i} > b}\dfrac{64\log{(T\dfrac{\Delta_{i}^{2}}{\sqrt{\log (K)}})}}{\Delta_{i}}
	\end{align*}		 
	 

	Lastly we can bound the error terms as, 
	\begin{align*}
	\sum\limits_{i\in A_{s^{*}}:0 < \Delta_{i}\leq b}\bigg(\dfrac{T^{1-\rho_{a}}2^{2\rho_{a}+\frac{3}{2}}}{\psi^{\rho_{a}}\Delta_{i}^{4\rho_{a}-1}} \bigg)= \sum\limits_{i\in A_{s^{*}}:0 < \Delta_{i}\leq b}\dfrac{80\sqrt{\log (K)}}{\Delta_{i}}
	\end{align*}
	
	
	Similarly for the term,
	
	\begin{align*}
	\sum_{i\in A\setminus A_{s^*}: 0 < \Delta_{i} \leq b}\bigg(\dfrac{T^{1-\rho_{s}}2^{2\rho_{s}+\frac{3}{2}}}{(\psi^{\rho_{s}})\Delta_{i}^{4\rho_{s} -1}} \bigg)=\sum_{i\in A\setminus A_{s^*}: 0 < \Delta_{i} \leq b}\dfrac{80\sqrt{\log (K)}}{\Delta_{i}}
	\end{align*}	 
		
	
	So, the total gap dependent regret bound for using both arm and cluster elimination comes of as
	\begin{align*}
	& \sum_{i\in A_{s^{*}}:\Delta_{i} > b}\bigg\lbrace \dfrac{112\sqrt{\log (K)}}{\Delta_{i}} + \Delta_{i} + \dfrac{64\log{(T\dfrac{\Delta_{i}^{2}}{\sqrt{\log (K)}})}}{\Delta_{i}} \bigg\rbrace + \sum_{i\in A:\Delta_{i} > b}\bigg\lbrace\dfrac{112\sqrt{\log (K)}}{\Delta_{i}} + 2\Delta_{i} \\
%%%%%%%%%%%%%%%%%%%%%	
	 &+  \dfrac{128\log{(T\dfrac{\Delta_{i}^{2}}{\sqrt{\log (K)}})}}{\Delta_{i}}\bigg\rbrace 
	  + \sum\limits_{i\in A_{s^{*}}:\Delta_{i} > b}\dfrac{80\sqrt{\log (K)}}{\Delta_{i}} + \sum\limits_{i\in A_{s^{*}}:0< \Delta_{i} <\leq b}\dfrac{80\sqrt{\log (K)}}{\Delta_{i}}\\
%%%%%%%%%%%%%%%%%%%%%
	& + \sum\limits_{i\in A\setminus A_{s^{*}}:\Delta_{i} > b}\dfrac{160\sqrt{\log (K)}}{\Delta_{i}}
	 + \sum\limits_{i\in A\setminus A \cup A_{s^{*}} :0 < \Delta_{i}\leq b}\dfrac{160\sqrt{\log (K)}}{\Delta_{i}} + \max\limits_{i\in A:\Delta_{i}\leq b}\Delta_{i}T 
	\end{align*}
\end{proof}




\section{Proof of Corollary 2}
\label{App:Proof:Corollary:2}
\begin{proof}
As stated in \cite{auer2010ucb}, we can have a bound on regret of the order of $\sqrt{KT\log K}$ in non-stochastic MAB setting. This is shown in Exp4\cite{auer2002nonstochastic} algorithm. Also we know from \cite{bubeck2011pure} that the function $x\in [0,1]\mapsto x\exp(-Cx^2)$ is  decreasing on $\left[\dfrac{1}{\sqrt{2C}},1\right ]$ for any $C>0$. So, taking $C=\left\lfloor \dfrac{14 T}{K}\right\rfloor$ and similarly, by choosing  $\Delta_{i}=\Delta=\sqrt{\dfrac{K\log K}{T}}>\sqrt{\dfrac{K}{14 T}}$ for all ${i:i\neq *}\in A$, in the bound of UCB1\cite{auer2002finite} we get,

\begin{align*}
\sum_{i:r_{i}<r^{*}}const \dfrac{\log T}{\Delta_{i}}=\dfrac{\sqrt{KT}\log T}{\sqrt{\log K}}
\end{align*}
So, this bound is worse than the non-stochastic setting and is clearly improvable and an upper bound regret of $\sqrt{KT}$ is possible as shown in \cite{audibert2009minimax} for MOSS which is consistent with the lower bound as proposed by Mannor and Tsitsiklis\cite{mannor2004sample}.

	Hence, we take $b\approx\sqrt{\dfrac{K\log K}{T}} > \sqrt{\dfrac{K}{14 T}}$(the minimum value for $b$), $\psi=\dfrac{T}{196 \log K}$, $\rho_{a}=\dfrac{1}{2}$ and $\rho_{s}=\dfrac{1}{2}$.
	
	Taking into account Theorem \ref{Result:Theorem:1} below, 
	
\begin{align*}
&\E [R_{T}]\leq 
\sum\limits_{\substack{i\in A_{s^{*}},\\\Delta_{i} > b}}\bigg\lbrace \dfrac{C_1(\rho_{a})T^{1-\rho_{a}}}{\Delta_{i}^{4\rho_{a}-1}} + \Delta_{i}
 + \frac{32\log{(\psi T\dfrac{\Delta_{i}^{4}}{16\rho_{a}^{2}})}}{\Delta_{i}} \bigg\rbrace
 + \! \! \sum\limits_{\substack{i\in A,\\\Delta_{i} > b}} \bigg\lbrace 2\Delta_{i}+
\dfrac{C_1(\rho_{s})T^{1-\rho_{s}}}{\Delta_{i}^{4\rho_{s}-1}} \\
&+ \frac{32\log{(\psi T\dfrac{\Delta_{i}^{4}}{16})}}{\Delta_{i}} 
+ \dfrac{32\log{(\psi T\dfrac{\Delta_{i}^{4}}{16})}}{\Delta_{i}}\bigg\rbrace 
%%%
+ \sum\limits_{\substack{i\in A_{s^{*}},\\ \Delta_{i} > b}} 
\frac{C_2(\rho_{a})T^{1-\rho_{a}}}{\Delta_{i}^{4\rho_{a}-1}}
+\sum\limits_{\substack{i\in A_{s^{*}},\\0 < \Delta_{i}\leq b}}\frac{C_2(\rho_a)T^{1-\rho_{a}}}{b^{4\rho_{a} -1}}\\ 
 & + \sum_{i\in A\setminus A_{s^*}: \Delta_{i} > b}\dfrac{2C_{2}(\rho_{s})T^{1-\rho_{s}}}{\Delta_{i}^{4\rho_{s}-1}} +\sum_{i\in A\setminus A_{s^*}: 0 < \Delta_{i} \leq b}\dfrac{2C_{2}(\rho_{s})T^{1-\rho_{s}}}{b^{4\rho_{s}-1}} +
 \!+\! \max\limits_{i:\Delta_{i}\leq b}\Delta_{i}T
\end{align*}

and putting the parameter values in the above Theorem \ref{Result:Theorem:1} result,	
	
	\begin{align*}
	\sum_{i\in A_{s^{*}}:\Delta_{i} > b}\bigg(\dfrac{T^{1-\rho_{a}}2^{1+4\rho_{a}}}{\psi^{\rho_{a}}\Delta_{i}^{4\rho_{a}-1}} \bigg)=& \bigg(K\dfrac{T^{1-\frac{1}{2}}2^{1+4\frac{1}{2}}}{p(\frac{T}{196 \log K})^{\frac{1}{2}}\Delta_{i}^{4\frac{1}{2}-1}} \bigg)=112\dfrac{\sqrt{KT}}{p}
	\end{align*}		
	 Similarly, for the term, 
	 \begin{align*}
	 \sum_{i\in A:\Delta_{i} > b}\bigg(\dfrac{T^{1-\rho_{s}}2^{1+4\rho_{s}}}{\psi^{\rho_{s}}\Delta_{i}^{4\rho_{s}-1}} \bigg) = 112\sqrt{KT}
	 \end{align*}
	 
	
	For the term regarding number of pulls,
	\begin{align*}
	\sum_{i\in A:\Delta_{i} > b}\dfrac{32\log{(\psi T\dfrac{\Delta_{i}^{4}}{16})}}{\Delta_{i}} &\leq  \dfrac{32K\sqrt{T}\log{(T^{2}\dfrac{K^{4}(\log K)^{2}}{T^{2}\log K})}}{\sqrt{K\log K}} \leq  \dfrac{64\sqrt{KT}\log{(K^{2}(\sqrt{\log K}))}}{\sqrt{\log K}}\\
	%%%%%%%%%%%%%%%%%%%%%%%
	&\leq 128\sqrt{KT\log K} + \dfrac{64\sqrt{KT}\log{(\sqrt{\log K})}}{\sqrt{\log K}}
	\end{align*}		
	
	Similarly for the term,
	\begin{align*}
	\sum_{i\in A:\Delta_{i} > b}\dfrac{32\log{(\psi T\dfrac{\Delta_{i}^{4}}{16})}}{\Delta_{i}} \leq 128\sqrt{KT\log K} + \dfrac{64\sqrt{KT}\log{(\sqrt{\log K})}}{\sqrt{\log K}}
	\end{align*}		
	
 	Lastly we can bound the error terms as, 
	\begin{align*}
	\sum\limits_{i\in A_{s^{*}}:0\leq\Delta_{i}\leq b}\bigg(\dfrac{T^{1-\rho_{a}}2^{2\rho_{a}+\frac{3}{2}}}{\psi^{\rho_{a}}\Delta_{i}^{4\rho_{a}-1}} \bigg)=\dfrac{K}{p}\bigg(\dfrac{T^{1-\frac{1}{2}}2^{2\frac{1}{2}+\frac{3}{2}}}{{(\frac{T}{196 \log K})^{\frac{1}{2}}}{(\Delta_{i})^{4*\frac{1}{2}-1}}} \bigg) < \dfrac{300 \sqrt{KT\log K} }{p}
	\end{align*}	 	
 	Similarly for the term,
 	\begin{align*}
 	\sum_{i\in A\setminus A_{s^*}: \Delta_{i} > b}\bigg(\dfrac{T^{1-\rho_{s}}2^{2\rho_{s}+\frac{3}{2}}}{(\psi^{\rho_{s}})\Delta_{i}^{4\rho_{s} -1}} \bigg) < 300(K-\dfrac{K}{p})\sqrt{\dfrac{T}{K\log K}}
	\end{align*} 	
	Also, for all $b\geq \sqrt{\dfrac{K}{14T}}$,
	\begin{align*}
 	\sum_{i\in A\setminus A_{s^*}: 0 < \Delta_{i} \leq b}\bigg(\dfrac{T^{1-\rho_{s}}2^{2\rho_{s}+\frac{3}{2}}}{(\psi^{\rho_{s}})b^{4\rho_{s} -1}} \bigg) < 300(K-\dfrac{K}{p})\sqrt{\dfrac{T\log K}{K}}
	\end{align*} 	
	
	Now, $K-\dfrac{K}{p}= K\left( \dfrac{p-1}{p} \right) < K\left(  \dfrac{\frac{K}{\log K}+1-1}{\frac{K}{\log K}+1 }\right) < \dfrac{K^2}{K+\log K}$. So, after putting the value of $p=\left\lceil \dfrac{K}{\log K} \right\rceil$, we get,
	
	\begin{align*}
	\E[R_{T}]\leq & 112\dfrac{\sqrt{T}\log K}{\sqrt{K}} + 128\dfrac{\sqrt{T}\log K}{\sqrt{K}} + \dfrac{64\sqrt{T\log K}\log{(\log K)}}{\sqrt{K}} + 112\sqrt{KT} + 128\sqrt{KT\log K}\\
	& + \dfrac{128\sqrt{KT}\log{(\log K)}}{\sqrt{\log K}} + \dfrac{300 \sqrt{T}\log K^{\frac{3}{2}} }{\sqrt{K}} + \dfrac{300 \sqrt{T}\log K}{\sqrt{K}} + 600 \dfrac{K}{K+\log K}\sqrt{KT\log K}\\
	%%%%%%%%%%%%%%%%%%%%
	& + 600 \dfrac{K}{K+\log K}\sqrt{KT}
	\end{align*}
 	
	So, the total bound for using both arm and cluster elimination cannot be worse than,
	
	\begin{align*}
	\E[R_{T}]\leq & 540\dfrac{\sqrt{T}\log K}{\sqrt{K}} + \dfrac{64\sqrt{T\log K}\log{(\log K)}}{\sqrt{K}} + 112\sqrt{KT} + 256\sqrt{KT\log K}\\
	& + \dfrac{128\sqrt{KT}\log{(\log K)}}{\sqrt{\log K}} + \dfrac{300 \sqrt{T}\log K^{\frac{3}{2}} }{\sqrt{K}} + 600 \dfrac{K}{K+\log K}\sqrt{KT\log K} + 600 \dfrac{K}{K+\log K}\sqrt{KT}\\ 
	\end{align*}		
\end{proof}

\section{Why Clustering?}
\label{App:E}

In this section we want to specify the apparent use of clustering. The error bounds are shown in Table \ref{App:E:table:3}.


%The regret bound for both arm elimination and pulls and error bound when a sub-optimal arm or a sub-optimal cluster eliminates ${*}$ or $s^{*}$ is given in Table \ref{App:E:table:2}, \ref{App:E:table:3}respectively.
 
%\begin{table}
%\caption{Regret Bound on Arm Elimination and Pulls}
%\label{App:E:table:2}
%\begin{center}
%\begin{tabular}{p{1.4cm}p{10.2cm}p{3.5cm}}
%\multicolumn{1}{c}{\bf Elim Type} &\multicolumn{1}{c}{\bf Regret Bound on Arm Elimination and Pulls} &\multicolumn{1}{c}{\bf Remarks} \\
%\hline \\
%Only Arm Elimination (ClusUCB-AE)	& \begin{align*}\sum_{i\in A:\Delta_{i} > b}\bigg\lbrace\underbrace{\bigg(\dfrac{C_{1}(\rho_{a})T^{1-\rho_{a}}}{\Delta_{i}^{4\rho_{a}-1}}\bigg)}_{\text{Case a1, Proposition \ref{proofTheorem:Prop:1}}} + \underbrace{\bigg(\Delta_{i}+ \dfrac{32\rho_{a}\log{(\psi  T\dfrac{\Delta_{i}^{4}}{16\rho_{a}^{2}})}}{\Delta_{i}}\bigg)}_{\text{case b1, Proposition \ref{proofTheorem:Prop:1}}}\bigg\rbrace \end{align*} & For $\rho_{a}=\frac{1}{4}$ and $\psi=K^{2}T$ this gives $ 2\sqrt{KT} + 32\sqrt{KT\log K} + \dfrac{16\log(\log K)}{\sqrt{\log K}}$. Hence the order is given by $O(\sqrt{KT\log K})$.\\
%%$\sum_{i\in A:\Delta_{i}\geq b}\bigg\lbrace\underbrace{\bigg(\dfrac{2^{1+4\rho_{a}}\rho_{a}^{2\rho_{a}}T^{1-\rho_{a}}}{(\psi)^{\rho_{a}}\Delta_{i}^{4\rho_{a}-1}}\bigg)}_{\text{Case a1, Proposition \ref{proofTheorem:Prop:1}}} + \underbrace{\bigg(\Delta_{i}+ \dfrac{32\rho_{a}\log{(\psi  T\dfrac{\Delta_{i}^{4}}{16\rho_{a}^{2}})}}{\Delta_{i}}\bigg)}_{\text{case b1, Proposition \ref{proofTheorem:Prop:1}}}\bigg\rbrace$
%\hline\\
%Only Cluster Elimination (ClusUCB-CE)	&\begin{align*} \sum_{i\in A:\Delta_{i} > b}\bigg\lbrace\underbrace{\bigg(\dfrac{C_{1}(\rho_{s})T^{1-\rho_{s}}}{\Delta_{i}^{4\rho_{s}-1}}\bigg)}_{\text{Case a, Proposition \ref{proofTheorem:Prop:2}}} +  \underbrace{\bigg(2\Delta_{i} +\dfrac{64\rho_{s}\log{(\psi T\dfrac{\Delta_{i}^{4}}{16\rho_{s}^{2}})}}{\Delta_{i}}\bigg)}_{\text{Case b1+b3, Proposition \ref{proofTheorem:Prop:2}}}\bigg\rbrace \end{align*} & With $\rho_{s}=\frac{1}{4}$ and  $\psi=K^{2}T$ this gives $ 2\sqrt{KT} + 64\sqrt{KT\log K} + \dfrac{32\log(\log K)}{\sqrt{\log K}}$. Hence, this is larger bound than using only arm elimination though the order is same $O(\sqrt{KT\log K})$.\\
%%$ \sum_{i\in A:\Delta_{i}\geq b}\bigg\lbrace\underbrace{\bigg(\dfrac{2^{2+4\rho_{s}}\rho_{s}^{2\rho_{s}}T^{1-\rho_{s}}}{(\psi)^{\rho_{s}}\Delta_{i}^{4\rho_{s}-1}}\bigg)}_{\text{Case a1+a2, Proposition \ref{proofTheorem:Prop:2}}} +  \underbrace{\bigg(\Delta_{i} +\dfrac{32\rho_{s}\log{(\psi T\dfrac{\Delta_{i}^{4}}{16\rho_{s}^{2}})}}{\Delta_{i}}\bigg)}_{\text{Case b1, Proposition \ref{proofTheorem:Prop:2}}}\bigg\rbrace$
%\hline\\
%Arm \& Cluster Elimination (ClusUCB) 	&\begin{align*} \sum_{i\in A_{s^{*}}:\Delta_{i} > b} \underbrace{\bigg(\dfrac{C_{1}(\rho_{a})T^{1-\rho_{a}}}{\Delta_{i}^{4\rho_{a}-1}}\bigg)}_{\text{Case a1, Arm Elim, Theorem \ref{Result:Theorem:1}}} +  \sum_{i\in A:\Delta_{i} > b} \bigg\lbrace\underbrace{\bigg(\dfrac{C_{2}(\rho_{s})T^{1-\rho_{s}}}{\Delta_{i}^{4\rho_{s}-1}}\bigg)}_{\text{Case a2, Clus Elim, Theorem \ref{Result:Theorem:1}}} 
% \\ + \underbrace{\bigg(\Delta_{i}+\dfrac{64\rho_{a}\log{(\psi T\dfrac{\Delta_{i}^{4}}{16\rho_{a}^{2}})}}{\Delta_{i}}\bigg)}_{\text{Case b1+b4, Arm Elim, Theorem \ref{Result:Theorem:1}}} +\underbrace{\bigg(\Delta_{i}+\dfrac{32\rho_{s}\log{(\psi T\dfrac{\Delta_{i}^{4}}{16\rho_{s}^{2}})}}{\Delta_{i}}\bigg)}_{\text{Case b1, Clus Elim, Theorem \ref{Result:Theorem:1}}}\bigg\rbrace\end{align*} & With $\rho_{a}=\frac{1}{4}$,$\rho_{s}=\frac{1}{2}$ and $\psi=K^{2}T$ this gives $\bigg\lbrace \dfrac{2\sqrt{KT}}{p} + 4\sqrt{\dfrac{T}{K\log K}} + 128\sqrt{KT\log K} + \dfrac{64\log(\log K)}{\sqrt{\log K}} \bigg\rbrace$. This is larger than the previous $2$ bounds though the order is same $O(\sqrt{KT\log K})$.
%%$\sum_{i\in A:\Delta_{i}\geq b} \bigg\lbrace \underbrace{\bigg(\dfrac{2^{1+4\rho_{a}}\rho_{a}^{2\rho_{a}}T^{1-\rho_{a}}}{(\psi)^{\rho_{a}}\Delta_{i}^{4\rho_{a}-1}}\bigg)}_{\text{Case a, Arm Elim, Theorem \ref{Result:Theorem:1}}} + \underbrace{\bigg(\dfrac{2^{2+4\rho_{s}}\rho_{s}^{2\rho_{s}}T^{1-\rho_{s}}}{(\psi)^{\rho_{s}}\Delta_{i}^{4\rho_{s}-1}}\bigg)}_{\text{Case a, Clus Elim, Theorem \ref{Result:Theorem:1}}} $   $+ \underbrace{\bigg(\Delta_{i}+\dfrac{32\rho_{a}\log{(\psi T\dfrac{\Delta_{i}^{4}}{16\rho_{a}^{2}})}}{\Delta_{i}}\bigg)}_{\text{Case b1, Arm Elim, Theorem \ref{Result:Theorem:1}}} +\underbrace{\bigg(\Delta_{i}+\dfrac{32\rho_{s}\log{(\psi T\dfrac{\Delta_{i}^{4}}{16\rho_{s}^{2}})}}{\Delta_{i}}\bigg)}_{\text{Case b1, Clus Elim, Theorem \ref{Result:Theorem:1}}}\bigg\rbrace$
%\end{tabular}
%\end{center}
%\end{table}

\begin{table}
\caption{Error Bound}
\label{App:E:table:3}
\begin{center}
\begin{tabular}{p{1.4cm}p{10.3cm}p{3.5cm}}
\multicolumn{1}{c}{\bf Elim Type} &\multicolumn{1}{c}{\bf Error Bound} &\multicolumn{1}{c}{\bf Remarks} \\
\hline \\
Only Arm Elimination (EClusUCB-AE)	& \begin{align*}\underbrace{\sum_{i\in A:\Delta_{i} > b}\bigg(\dfrac{C_{2}(\rho_{a})T^{1-\rho_{a}}}{\Delta_{i}^{4\rho_{a} -1}} \bigg)}_{\text{Case b2, Proposition \ref{proofTheorem:Prop:1}}} + \underbrace{\sum_{i\in A:0 < \Delta_{i}\leq b}\bigg( \dfrac{C_{2}(\rho_{a})T^{1-\rho_{a}}}{b^{4\rho_{a} -1}} \bigg)}_{\text{Case b2, Proposition \ref{proofTheorem:Prop:1}}}\end{align*}  & With $\rho_{a}=\frac{1}{2},$ and $\psi=\frac{T}{196 \log K}$ this gives $300\sqrt{KT}+300\sqrt{KT\log K}$. Hence, this has an order of $O(\sqrt{KT\log K})$.\\
%$\underbrace{\sum_{i\in A:\Delta_{i}\geq b}\bigg(\dfrac{T^{1-\rho_{a}}\rho_{a}^{2\rho_{a}}2^{2\rho_{a}+\frac{3}{2}}}{(\psi)^{\rho_{a}}\Delta_{i}^{4\rho_{a} -1}} \bigg)}_{\text{Case b2, Proposition \ref{proofTheorem:Prop:1}}} + \underbrace{\sum_{i\in A:0\leq\Delta_{i}\leq b}\bigg( \dfrac{T^{1-\rho_{a}}\rho_{a}^{2\rho_{a}}2^{2\rho_{a}+\frac{3}{2}}}{(\psi)^{\rho_{a}}b^{4\rho_{a} -1}} \bigg)}_{\text{Case b2, Proposition \ref{proofTheorem:Prop:1}}}$
\hline\\
%%%%%%%%%%%%%%%%%%%%%%%%%%%%%%%%%%%%%%%%%%%%%%%%%%%%%%%%%%%%%%%%%%%%%%%%
%Only Cluster Elimination	 (ClusUCB-CE) & \begin{align*} \underbrace{\sum_{i\in A\setminus A_{s^*}:\Delta_{i} > b}\bigg(\dfrac{2C_{2}(\rho_{s})T^{1-\rho_{s}}}{\Delta_{i}^{4\rho_{s} -1}} \bigg)}_{\text{Case b2+b3, Proposition \ref{proofTheorem:Prop:2}}} +\underbrace{\sum_{i\in A\setminus A_{s^*}:0\leq\Delta_{i}\leq b}\bigg(\dfrac{2C_{2}(\rho_{s})T^{1-\rho_{s}}}{b^{4\rho_{s} -1}} \bigg)}_{\text{Case b2+b3, Proposition \ref{proofTheorem:Prop:2}}} \end{align*} & With $\rho_{s}=\frac{1}{2}$ and $\psi=K^{2}T$ this gives $4\sqrt{KT}+4\sqrt{KT\log K}$. This is same as the bound using only arm elimination and has an order of $O(\sqrt{KT\log K})$.\\
%\hline\\
%%%%%%%%%%%%%%%%%%%%%%%%%%%%%%%%%%%%%%%%%%%%%%%%%%%%%%%%%%%%%%%%%%%%%%%%%%
Arm \& Cluster Elimination (EClusUCB) 	& \begin{align*}  \underbrace{\sum_{i\in A_{s^{*}}:\Delta_{i} > b}\bigg(\dfrac{C_{2}(\rho_{a})T^{1-\rho_{a}}}{\Delta_{i}^{4\rho_{a}-1}} \bigg)+ \sum_{i\in A_{s^{*}}:0\leq\Delta_{i}\leq b}\bigg(\dfrac{C_{2}(\rho_{a})T^{1-\rho_{a}}}{b^{4\rho_{a} -1}} \bigg)}_{\text{Case b2, Arm Elim, Theorem \ref{Result:Theorem:1}}}\\   
 + \underbrace{\sum_{i\in A\setminus A_{s^*}:\Delta_{i} > b}\bigg(\dfrac{2C_{2}(\rho_{s})T^{1-\rho_{s}}}{\Delta_{i}^{4\rho_{s}-1}} \bigg)+ \sum_{i\in A\setminus A_{s^*}:0\leq\Delta_{i}\leq b}\bigg(\dfrac{2C_{2}(\rho_{s})T^{1-\rho_{s}}}{b^{4\rho_{s} -1}} \bigg)}_{\text{Case b3+b4, Clus Elim, Theorem \ref{Result:Theorem:1}}} \end{align*} & With $\rho_{a}=\frac{1}{2}$, $\rho_{s}=\frac{1}{2}, p=\lceil \frac{K}{\log K}\rceil$ and $\psi=\frac{T}{196 \log K}$ this gives $\frac{300 \sqrt{T}\log K^{\frac{3}{2}} }{\sqrt{K}} + \frac{300 \sqrt{T}\log K}{\sqrt{K}} + 600 \frac{K}{K+\log K}\sqrt{KT\log K} + 600 \frac{K}{K+\log K}\sqrt{KT}$. So we can reduce the error bound to $O(\frac{K}{K+\log K}\sqrt{KT\log K})$.\\
\hline
\end{tabular}
\end{center}	
\end{table}
% From Table \ref{App:E:table:2} we can see that from the definition of $\rho_{s},\rho_{a}$ we can have a regret bound for jointly doing the arm and cluster elimination of the same order as doing arm elimination or cluster elimination alone. 
 
While looking at the error terms in Table~\ref{App:E:table:3}, we see that using just arm elimination (EClusUCB-AE) the elimination error bound is more than using both arm and cluster  elimination simultaneously (EClusUCB). 
%From Table \ref{App:E:table:3}, we can see that the error term for using both arm and cluster elimination can become low depending on how we choose $p$ since $|A_{s^{*}}|\leq \lceil\frac{A}{p}\rceil$ and in Corollary \ref{Result:Corollary:2} we proved that taking $p=\big\lceil \frac{K}{\log K} \big\rceil$ reduces the elimination error bound.
 
 
%%%%%%%%%%%%%%%%%%%%
%Table for Regret Bounds  
%%%%%%%%%%%%%%%%%%%% 
 
%\begin{figure}
%\centering
%  \begin{tabular}{c}
%  \begin{subfigure}{0.45\textwidth}
% \tabl{c}{\scalebox{0.8}{\begin{tikzpicture}
%      \begin{axis}[
%	xlabel={timestep},
%	ylabel={Cumulative regret},
%       clip mode=individual,grid,grid style={gray!30},
%  legend style={at={(0.5,-0.2)},anchor=north, legend columns=2} ]
%      % UCB
%\addplot table[x index=0,y index=1,col sep=tab,each nth point={10}] {results/Expt4/clUCB1comp_subsampled.txt};
%\addplot table[x index=0,y index=1,col sep=tab,each nth point={10}] {results/Expt4/clUCB2comp_subsampled.txt};
%\addplot table[x index=0,y index=1,col sep=tab,each nth point={10}] {results/Expt4/clUCB3comp_subsampled.txt};
%\addplot table[x index=0,y index=1,col sep=tab,each nth point={10}] {results/Expt4/MOSScomp_subsampled.txt};
%\addplot table[x index=0,y index=1,col sep=tab,each nth point={10}] {results/Expt4/clUCB4comp_subsampled.txt};
%\addplot table[x index=0,y index=1,col sep=tab,each nth point={10}] {results/Expt4/clUCB5comp_subsampled.txt};
%      \legend{ClusUCB(1A),ClusUCB(4B),ClusUCB(10B),MOSS,ClusUCB(5S),ClusUCB(10S)}
%      %\legend{ClusUCB (NC, p=1),ClusUCB (C, p=4),ClusUCB(C, p=10) ,MOSS, ClusUCB(C, p=5, NAE), ClusUCB(C, p=10, NAE)}
%      %\legend{ClusUCB(1,A),ClusUCB(4,B),ClusUCB(10,B), MOSS,ClusUCB(5,S), ClusUCB(10,A)}
%      \end{axis}
%      \end{tikzpicture}}\\}
%			\caption{Experiment $4$: ClusUCB for various $p$. ClusUCB(1A)= $\lbrace$ p=$1$,Only Arm Elimination $\rbrace$, ClusUCB(4B)=$\lbrace$ p=$4$, Both Arm and Cluster Elimination$\rbrace$, ClusUCB(5S)=$\lbrace$ p=$5$, Only Cluster Elimination$\rbrace$. }
%  \label{Fig:variousClus}
%  \end{subfigure}
%  \end{tabular}
%
%\end{figure}

%\end{remark}

%\section{Regret Bound Table}
%\label{App:F}
%The regret upper bound of various algorithms are given in Table \ref{sample-table}.
%\begin{table}
%\caption{Regret Upper Bound of Algorithms}
%\label{sample-table}
%\begin{center}
%\begin{tabular}{l|l}
%\multicolumn{1}{c}{\bf Algorithm}  &\multicolumn{1}{c}{\bf Regret Upper Bound} \\
%\hline \\
%UCB1         &\hspace*{5em}$\min\bigg\lbrace O(\sqrt{KT\log T}) ,O\bigg(\dfrac{K\log T}{\Delta}\bigg)\bigg\rbrace$ \\
%%UCB2         &\hspace*{5em}$O\bigg(K\bigg(\dfrac{(1 + \epsilon(\alpha)) log(T)}{2\Delta} + C(\alpha)\bigg)\bigg)$, $0<\alpha<1$ \\
%$\epsilon_{n}$-greedy         &\hspace*{5em}$O\bigg(\dfrac{K\Delta\log T}{d^{2}}\bigg)$, $0<d<\Delta$ \\
%%EXP3             &\hspace*{5em}$O\bigg(S \sqrt{KT \log(KT)}\bigg)$, where $S$ is the hardness of the problem \\
%%UCB($\delta$)	&\hspace*{5em}$O\bigg(\dfrac{16K}{\Delta}\log\big(\dfrac{2K}{\Delta\delta}\big)\bigg)$ , where $\delta$ is the error probability\\
%UCB-Improved             &\hspace*{5em}$\min\bigg\lbrace O\bigg(\sqrt{KT\log K}\bigg), O\bigg(\dfrac{K\log (T\Delta^{2})}{\Delta}\bigg)\bigg\rbrace$ \\
%MOSS				&\hspace*{5em}$\min\bigg\lbrace O\bigg(\sqrt{KT}\bigg), O\bigg(\dfrac{K\log(T\Delta^{2}/K)}{\Delta}\bigg) \bigg \rbrace$\\
%%KL-UCB         &\hspace*{5em}$O\bigg(K\bigg(\dfrac{\Delta \log(T)(1 + \epsilon)}{d(r_{i}, r^{*} )} + \log(\log(T)) + \dfrac{(\epsilon)}{T^{\beta(\epsilon)}}\bigg)\bigg)$, where $\epsilon > 0$ and $d(r_{i}, r^{*})>2\Delta_{i}^{2}$\\
%UCB-Clustered             &\hspace*{5em}$\min\bigg\lbrace O\bigg(\sqrt{KT\log K}\bigg),O\bigg(\dfrac{K\log\big (\dfrac{T\Delta^{2}}{\sqrt{\log (KT)}}\big)}{\Delta}\bigg)\bigg\rbrace$\\
%\end{tabular}
%\end{center}
%\end{table}

%%%%%%%%%%%%%%%%%%%%%%%%%%%%%%%%
%Moved to Main paper
%%%%%%%%%%%%%%%%%%%%%%%%%%%%%%%%

%\section{Efficient Clustered UCB}

%\begin{algorithm}[t]
%\caption{EClusUCB}
%\label{alg:eclusucb}
%\begin{algorithmic}
%\State {\bf Input:} Number of clusters $p$, time horizon $T$, exploration parameters $\rho_a$, $\rho_s$ and $\psi$.
%\State {\bf Initialization:} Set $m:=0$, $B_{0}:=A$, $S_0 = S$, $\epsilon_{0}:=1$, $M=\big \lfloor \dfrac{1}{2}\log_{2} \dfrac{7T}{K}\big\rfloor$, $n_{0}=\bigg\lceil\dfrac{2\log{(\psi T\epsilon_{0}^{2})}}{\epsilon_{0}}\bigg\rceil$ and  $N_{0}=K*n_{0}$.
%\State Create a partition $S_0$ of the arms at random into $p$ clusters of size up to $\ell=\bigg\lceil \dfrac{K}{p} \bigg\rceil$ each.
%\State Pull each arm once
%\For{$t=K+1,..,T$}	
%\State Pull arm $i$ in $B_m$ such that $\max_{i\in B_{m}}\bigg\lbrace \hat{r}_{i} + \sqrt{\dfrac{\rho_{s}\log{(\psi T\epsilon_{m}^{2})}}{2 n_{i}}} \bigg\rbrace$
%\State $t:=t+1$
%\ArmElim
%\State For each cluster $s_k \in S_{m}$, delete arm ${i}\in s_{k}$ from $B_{m}$ if
%\begin{align*}
%\hat{r}_{i} + \sqrt{\dfrac{\rho_{a}\log{(\psi T\epsilon_{m}^{2})}}{2 n_{i}}}  < \max_{{j}\in s_{k}}\bigg\lbrace\hat{r}_{j} -\sqrt{\dfrac{\rho_{a}\log{(\psi T\epsilon_{m}^{2})}}{2 n_{j}}} \bigg\rbrace
%\end{align*}
% where $\rho_{a}=\dfrac{1}{w_{m}}$ and remove all such arms from $B_{m}$.
%\EndArmElim
%\ClusElim
%\State Delete cluster $s_{k}\in S_{m}$ and remove all arms $i\in s_{k}$ from $B_{m}$ if 
%\begin{align*}
% \max_{{i}\in s_{k}}\bigg\lbrace\hat{r}_{i} + \sqrt{\dfrac{\rho_{s}\log{(\psi T\epsilon_{m}^{2})}}{2 n_{i}}}\bigg\rbrace 
% < \max_{{j}\in B_{m}} \bigg\lbrace\hat{r}_{j} - \sqrt{\dfrac{\rho_{s} \log{(\psi T\epsilon_{m}^{2})}}{2 n_{j}}}\bigg\rbrace.
%\end{align*}
%  and remove all such arms in the cluster $s_{k}$ from $B_{m}$ to obtain $B_{m+1}$.
%\EndClusElim
%
%\If{$t\geq N_{m}$ and $m\leq M$}
%\ResParam
%\State $\epsilon_{m+1}:=\dfrac{\epsilon_{m}}{2}$\vspace{0.5ex}
%\State $B_{m+1}:=B_{m}$
%\State $n_{m}:=\bigg\lceil\dfrac{2\log{(\psi T\epsilon_{m}^{2})}}{\epsilon_{m}}\bigg\rceil$
%\State $N_{m}:=t+|B_{m}| * n_{m}$
%\State $m:=m+1$
%\EndResParam
%\State \hspace*{2em} $\ell_{m+1}:=\min\lbrace 2\ell_{m}, K\rbrace$
%\State \hspace*{2em} $w_{m+1}:=2w_{m}$
%\State Stop if $|B_{m}|=1$ and pull ${i}\in B_{m}$ till $T$ is reached.
%\EndIf
%\EndFor
%\end{algorithmic}
%\end{algorithm}
%
%In ClusUCB we modified UCB-Improved to reduce early exploration and do faster elimination of sub-optimal arms and clusters with the help of clustering. With careful choosing of exploration regulatory factor $\psi$ and elimination parameters $\rho_{a}$ and $\rho_{s}$ we were able to bring down the cumulative regret compared to UCB-Improved. One of the principal dis-advantage that still remain is that ClusUCB is a round based algorithm which samples all remaining arms equally in each round (still less compared to UCB-Improved). In this section, we introduce a further modification,  as shown in Algorithm \ref{alg:eclusucb} called Efficient Clustered UCB hence referred to as EClusUCB where we introduce the idea of optimistic greedy sampling. A similar idea was also used in \cite{liu2016modification} which they used to modify the UCB-Improved algorithm. We further modify the idea by introducing clustering and arm elimination parameters. Also we use different exploration regulatory factor and we come up with a cumulative regret bound whereas  \cite{liu2016modification} only gives simple regret bound. In optimistic greedy sampling we only sample the arm with the highest upper confidence bound in each timestep. Also in EClusUCB we check arm elimination condition and cluster elimination condition in every timestep and update parameters when a round is complete. Drawing from the idea of UCB-Improved, we divide each round into $|B_{m}|n_{m}$ timesteps so that each surviving arms can be allocated atmost $n_{m}$ pulls. In the next section we analyze the regret of EClusUCB. 


%%%%%%%%%%%%%%%%%%%%%%%%%%%%%%%%%%%%%%%%%%%%%%%%%%%%%%%%%
% Proof of EClusUCB
%%%%%%%%%%%%%%%%%%%%%%%%%%%%%%%%%%%%%%%%%%%%%%%%%%%%%%%%%
%\section{Proof of Theorem 2}
%\label{App:EClusUCB}
%
%\begin{theorem}[\textbf{\textit{Regret bound}}]
%\label{Result:Theorem:2}
%The regret $R_T$ of EClusUCB satisfies
%\begin{align*}
%&\E [R_{T}]\leq 
%\sum\limits_{\substack{i\in A_{s^{*}},\\\Delta_{i} > b}}\bigg\lbrace \dfrac{C_1(\rho_{a})T^{1-\rho_{a}}}{\Delta_{i}^{4\rho_{a}-1}} + \Delta_{i}
% + \frac{32\rho_{a}\log{(\psi T\dfrac{\Delta_{i}^{4}}{16\rho_{a}^{2}})}}{\Delta_{i}} \bigg\rbrace
% + \! \! \sum\limits_{\substack{i\in A,\\\Delta_{i} > b}} \bigg\lbrace 2\Delta_{i}+
%\dfrac{C_1(\rho_{s})T^{1-\rho_{s}}}{\Delta_{i}^{4\rho_{s}-1}} \\
%&+ \frac{32\rho_{a}\log{(\psi T\dfrac{\Delta_{i}^{4}}{16\rho_{a}^{2}})}}{\Delta_{i}} 
%+ \dfrac{32\rho_{s}\log{(\psi T\dfrac{\Delta_{i}^{4}}{16\rho_{s}^{2}})}}{\Delta_{i}}\bigg\rbrace
%+ \sum\limits_{\substack{i\in A_{s^{*}},\\ \Delta_{i} > b}} 
%\frac{C_2(\rho_{a})T^{1-\rho_{a}}}{\Delta_{i}^{4\rho_{a}-1}}
%+\sum\limits_{\substack{i\in A_{s^{*}},\\0 < \Delta_{i}\leq b}}\frac{C_2(\rho_a)T^{1-\rho_{a}}}{b^{4\rho_{a} -1}}\\ 
%%%%%%%%
%%&+ \!\sum\limits_{\substack{i\in A,\\ \Delta_{i} > b}}\! \! \dfrac{C_2(\rho_{s})T^{1-\rho_{s}}}{\Delta_{i}^{4\rho_{s}-1}}
%% + \!\sum\limits_{\substack{i\in A,\\0 < \Delta_{i}\leq b}}\! \! \dfrac{C_2(\rho_{s})T^{1-\rho_{s}}}{b^{4\rho_{s} -1}}  \\
%&+ \sum_{\substack{i\in A\setminus A_{s^*}:\\\Delta_{i}> b}}\dfrac{2C_{2}(\rho_{s})T^{1-\rho_{s}}}{\Delta_{i}^{4\rho_{s}-1}} +\sum_{\substack{i\in A \setminus A_{s^*}:\\ 0 < \Delta_{i} \leq b}}\dfrac{2C_{2}(\rho_{s})T^{1-\rho_{s}}}{b^{4\rho_{s}-1}}\\
%& \!+\! \max\limits_{i:\Delta_{i}\leq b}\Delta_{i}T, 
%\end{align*}
%
%where $b\geq \sqrt{\frac{K}{14 T}}$, $C_1(x) = \frac{2^{1+4x}x^{2x}}{\psi^{x}}$, $C_2(x) = \frac{2^{2x+\frac{3}{2}}x^{2x}}{\psi^{x}}$ and $A_{s^{*}}$ is the subset of arms in cluster $s^{*}$ containing optimal arm $a^{*}$.
%%, $\rho_{a}=\dfrac{1}{2},\rho_{s}=\dfrac{1}{2}$ and $\psi=K^{2}T$.
%\end{theorem}
%
%We can see from the result of this theorem that the regret upper bound for EClusUCB is same as ClusUCB. Also we can specialize the result of this Theorem \ref{Result:Theorem:2} using Corollary \ref{Result:Corollary:1} and \ref{Result:Corollary:2} to achieve a similar bound as for Theorem \ref{Result:Theorem:1}.
%
%\begin{proof}
%\label{sec:proofTheorem1}
%%$\Delta_{i}^{'}=r_{a_{\max_{s_{k}}}} - r_{i}$ such that $a_{i}\in s_{k}$,
%% m_{i}^{'}=\min{\lbrace m|\sqrt{\rho_{a}\epsilon_{m}} < \frac{\Delta_{i}^{'}}{2} \rbrace}
%Let $A^{'}=\lbrace i \in A,\Delta_{i}> b\rbrace$,  $A^{''}=\lbrace i \in A, \Delta_{i} > 0\rbrace$, $A^{'}_{s_{k}}=\lbrace i \in A_{s_{k}},\Delta_{i}> b\rbrace$ and $A^{''}_{s_{k}}=\lbrace i \in A_{s_{k}}, \Delta_{i} > 0 \rbrace$. $C_{g}$ is the cluster set containing max payoff arm from each cluster in $g$-th round. The arm having the highest payoff in a cluster $s_{k}$ is denote by $a_{\max_{s_{k}}}$. Let for each sub-optimal arm ${i}\in A$, $m_{i}=\min{\lbrace m|\sqrt{\rho_{a}\epsilon_{m}} < \frac{\Delta_{i}}{2} \rbrace}$ and let for each cluster $s_{k}\in S$, $g_{s_{k}}=\min{\lbrace g|\sqrt{\rho_{s}\epsilon_{g}} < \frac{\Delta_{a_{\max_{s_{k}}}}}{2} \rbrace}$. 
%Let $\check{A}=\lbrace {i}\in A^{'} | {i}\in s_{k} , \forall s_{k}\in S \rbrace$. Also $z_{i}$ denotes total number of times an arm $i$ has been pulled from $1$ to $(t-1)$-th timestep. In the $m$-th round, $n_{m}$ denotes the number of pulls allocated to the surviving arms in $B_{m}$. The analysis proceeds by considering the contribution to the regret in each of the following cases:
%
%\textbf{Case a:} \textit{Some sub-optimal arm ${i}$ is not eliminated in round $\max(m_{i},g_{s_{k}})$ or before, with the optimal arm ${*}\in C_{\max(m_{i},g_{s_{k}})}$.}
%
%We consider an arbitrary sub-optimal arm ${i}$ and analyze the contribution to the regret when $i$ is not eliminated in the following exhaustive sub-cases:\\
%\textbf{Case a1:} \textit{In round $\max(m_{i},g_{s_{k}})$, ${i} \in s^{*}$.}
%
%This case is similar to Case a1, Theorem \ref{Result:Theorem:1}. Let $c_{i}=\sqrt{\dfrac{\rho_{a}\log{(\psi T\epsilon_{m}^{2})}}{2 z_{i}}}$. For an arm to get eliminated it must satisfy the following condition,
%\begin{align}
%\hat{r}_{i}  \le r_{i} + c_{i} \text{ and } 
%\hat{r}^{*}\geq  r^{*} - c^{*}, \label{eq:armelim-casea2}
%\end{align}
%
%Since each round now consist of $|B_{m}|n_{m}$ timesteps and since at every timestep the arm elimination condition is being checked, following a parallel arguement as in Case a1, Theorem $\ref{Result:Theorem:1}$ we can show that when $z_{i} = n_{m_{i}}$ then \\ $c_{i} \leq c_{m_{i}} \leq \sqrt{\dfrac{\rho_{a}\log{(\psi T\epsilon_{m}^{2})}}{2 n_{m_{i}}}} = \sqrt{\rho_{a}\epsilon_{m_{i}+1}} < \frac{\Delta_{i}}{4}$, since $z_{i} =  n_{m_{i}}=\frac{2\log{(\psi T\epsilon_{m_{i}}^{2})}}{\epsilon_{m_{i}}}$ and $\rho_{a}\in (0,1]$.
%
%Subsequently following the steps of Case a1, Theorem \ref{Result:Theorem:1} we can upper bound the probability of the complementary of the events in \ref{eq:armelim-casea2} and show that the probability that a sub-optimal arm ${i}$ is not eliminated in any round on or before $m_{i}$ is bounded above by  $\bigg(\frac{2}{(\psi T\epsilon_{m_{i}}^{2})^{\rho_{a}}}\bigg)$.
%
%%%%%%%%%%%%%%%%%%%%%%%%%%%%%%%%%%%%%%%%%%%%%%%%%%%%%%%%%%%%%%%%%%%%%%%%%%%%%%%%%%%%%%%%%%%%%%%%   
%\textbf{Case a2:} \textit{In round $\max(m_{i},g_{s_{k}})$, ${i} \in s_k$ for some $s_k \ne s^{*}$.}
%
%Let $c_{a_{\max_{s_{k}}}}=\sqrt{\dfrac{\rho_{s}\log{(\psi T\epsilon_{m}^{2})}}{2 z_{a_{max_{s_{k}}}}}}$ and $z_{a_{max_{s_{k}}}}$ is the number of times $a_{max_{s_{k}}}$ has been pulled. Following a parallel argument like in Case $a1$, we have to bound the following two events of arm $a_{\max_{s_{k}}}$ not getting eliminated on or before $g_{s_{k}}$-th round,
%\begin{align*}
%  \hat{r}_{a_{\max_{s_{k}}}} \geq r_{a_{\max_{s_{k}}}} +c_{a_{\max_{s_{k}}}} \text{ and } \hat{r}^{*} \leq r^{*} -c^{*}  
%\end{align*} 
%
%As we argued in previous case, since cluster elimination condition being verified in each timestep, the above two conditions are possible when $z_{a_{max_{s_{k}}}} = n_{g_{s_{k}}}$. We can prove using Chernoff-Hoeffding bounds and considering independence of events mentioned above, that for $c_{a_{\max_{s_{k}}}} \leq c_{g_{s_{k}}} \leq \sqrt{\frac{\rho_{s} \log (\psi T\epsilon_{g_{s_{k}}}^{2})}{2 n_{g_{s_{k}}}}}$ and  $n_{g_{s_{k}}}=\frac{2\log{(\psi T\epsilon_{g_{s_{k}}}^{2})}}{\epsilon_{g_{s_{k}}}}$ the probability of the above two events is bounded by $\bigg(\dfrac{2}{(\psi  T\epsilon_{g_{s_{k}}}^{2})^{\rho_{s}}}\bigg)$.
%%Summing, the two up, the probability that a sub-optimal cluster arm $a_{\max_{s_{k}}}\in C_{g_{s_{k}}}$ is not eliminated
%  Now, for any round $g_{s_{k}}$, all the elements of $C_{\max(m_{i},g_{s_{k}})}$ are the respective maximum payoff arms of their cluster $s_{k}, \forall s_{k}\in S$, and since clusters are fixed so we can bound the maximum probability that a sub-optimal arm ${i}\in A^{'}$  and ${i}\in s_{k}$ such that $a_{\max_{s_{k}}}\in C_{g_{s_{k}}}$ is not eliminated on or before the $g_{s_{k}}$-th round by the same probability as above. 
%
%Summing up over all $p$ clusters and bounding the regret for each arm $i\in A_{s_{k}}^{'}$ trivially by $T\Delta_{i}$ and following the steps of Case a2, Theorem \ref{Result:Theorem:1}, we can show that the regret for not eliminating clusters on or before round $g_{s_{k}}$ is upper bounded by,
%
%\begin{align*}
%\sum_{i\in A^{'}}\frac{C_{1}(\rho_{s})T^{1-\rho_{s}}}{\Delta_{i}^{4\rho_{s}-1}}
%\end{align*}
%
%
%
%Summing the bounds in Cases $a1-a2$ and observing that the bounds in the aforementioned cases hold for any round $C_{\max \lbrace m_i,g_{s_k}\rbrace}$, we obtain the following contribution to the expected regret from Case a:
%   %Taking summation of the events mentioned above($a1$-$a4$) gives us an upper bound on the regret given that the optimal arm $a^{*}$ is still surviving, 
%\begin{align*}
%&\sum_{i\in A_{s^*}} \frac{C_{1}(\rho_{a})T^{1-\rho_{a}}}{\Delta_{i}^{4\rho_{a}-1}} + \sum_{i\in A^{'}}\bigg(\frac{C_{1}(\rho_{s})T^{1-\rho_{s}}}{\Delta_{i}^{4\rho_{s}-1}}\bigg)
%\end{align*}
%
%
%
%%%%%%%%%%%%%%%%%%%%%%%%%%%%%%%%%%%%%%%%%%%%%%%%%%%%%%%%%%%%%%%%%%%%%%%%%%%%%%%%%%%%%%%%%%%%%
%\textbf{Case b:} \textit{For each arm $i$, either ${i}$ is eliminated in round $\max (m_{i},g_{s_{k}})$ or before or there is no optimal arm ${*}$ in $C_{\max(m_{i},g_{s_{k}})}$.} \\
%
%\textbf{Case b1:} \textit{${*}\in C_{\max(m_{i},g_{s_{k}})}$ for each arm $i \in A'$ and cluster $s_k \in \check A$.} 
%
%%\todos{define $\check A$}
%
%The condition in the case description above implies the following: \\
%\begin{inparaenum}[\bfseries (i)]
%\item each sub-optimal arm ${i}\in A^{'}$ is  eliminated on or before $\max (m_{i},g_{s_{k}})$ and hence  pulled not more than $z_{i}$ number of times. But according to the condition of Case a1, $z_{i} < n_{m_{i}}$ number of times.\\
%\item each sub-optimal cluster $s_k \in \check A$ is eliminated on or before $\max (m_{i},g_{s_{k}})$ and hence  pulled not more than $n_{a_{\max_{s_{k}}}}$ number of times. But again according to condition Case a2,  $z_{a_{\max_{s_{k}}}} < n_{g_{s_{k}}}$ number of times.
%\end{inparaenum}
%
%Hence, following the same steps as in Case b1, Theorem \ref{Result:Theorem:1}, the maximum regret suffered due to pulling of a sub-optimal arm or a sub-optimal cluster is no more than the following:
% \begin{align*}
%\sum_{i\in A^{'}}\!\bigg[ 2\Delta_{i}+\dfrac{32(\rho_{a}\log{(\psi T\dfrac{\Delta_{i}^{4}}{16\rho_{a}^{2}})} + \rho_{s}\log{(\psi T\dfrac{\Delta_{i}^{4}}{16\rho_{s}^{2}})})}{\Delta_{i}} \bigg]
% \end{align*}
%
% 
%%%%%%%%%%%%%%%%%%%%%%%%%%%%%%%%%%%%%%%%%%%%%%%%%%%%%%%%%%%%%%%%%%%%%%%%%%%%%%%%%%%%%%%%%%%%%%%%   
%%\textbf{Case b2:} \textit{Optimal arm $a^{*}$ is eliminated by a sub-optimal arm.}\\
%  %
%	%This, can happen in $3$ ways,
%%\newline
%\textbf{Case b2:} \textit{${*}$ is eliminated by some sub-optimal arm in $s^*$} \\
%%In this case, we are concerned with the arm elimination condition only. 
%Optimal arm $a^*$ can get eliminated by some sub-optimal arm $i$ only if arm elimination condition holds, i.e., 
%\begin{align*}
%\hat r_{i} - c_{i} > \hat{r}^{*}+ c^{*},
%\end{align*}
%where, as mentioned before, $c_{i} \leq c_{m_{i}} \leq \sqrt{\frac{\rho_{a}\log (\psi T\epsilon_{m_{i}}^{2})}{2 n_{m_{i}}}}$.
%From analysis in Case $a1$, notice that, if \eqref{eq:armelim-casea2} holds in conjunction with the above, arm $i$ gets eliminated. Also, recall from Case $a1$ that the events complementary to \eqref{eq:armelim-casea2} have low-probability and can be upper bounded by $\frac{2}{(\psi  T\epsilon_{m_{*}}^{2})^{\rho_{a}}}$. Moreover, a sub-optimal arm that eliminates $*$ has to survive until round $m_*$. In other words, 
%all arms ${j}\in s^{*}$ such that $m_{j} < m_{*}$ are eliminated on or before $m_*$ (this corresponds to case b1). 
%Let, the arms surviving till $m_{*}$ round be denoted by $A^{'}_{s^{*}}$. This leaves any arm $a_{b}$ such that $m_{b}\geq m_{*} $ to still survive and eliminate arm ${*}$ in round $m_{*}$. Let, such arms that survive ${*}$ belong to $A^{''}_{s^{*}}$. Also maximal regret per step after eliminating ${*}$ is the maximal $\Delta_{j}$ among the remaining arms in $A^{''}_{s^{*}}$ with $m_{j}\geq m_{*}$.  Let $m_{b}=\min\lbrace m|\sqrt{\rho_{a}\epsilon_{m}}<\frac{\Delta_{b}}{2}\rbrace$. Let $C_2(x) = \frac{2^{2x+\frac{3}{2}}x^{2x}}{\psi^{x}}$. Hence, the maximal regret after eliminating the arm ${*}$ is upper bounded by, 
%\begin{align*}
%&\sum_{m_{*}=0}^{max_{j\in A^{'}_{s^{*}}}m_{j}}\sum_{\substack{i\in A^{''}_{s^{*}}: \\ m_{i}\geq m_{*}}}\bigg(\dfrac{2}{(\psi  T\epsilon_{m_{*}}^{2})^{\rho_{a}}} \bigg).T\max_{\substack{j\in A^{''}_{s^{*}}: \\ m_{j}\geq m_{*}}}{\Delta}_{j}\\
%\end{align*}
%Therefore, following the similar steps as in Case b2, Theorem \ref{Result:Theorem:1}, we can show that the above regret is upper bounded by,
%\begin{align*}
%\sum_{i\in A^{'}_{s^{*}}}\dfrac{ C_{2}(\rho_{a}) T^{1-\rho_{a}}}{\Delta_{i}^{4\rho_{a}-1}} +\sum_{i\in A^{''}_{s^{*}}\setminus A^{'}_{s^{*}}}\dfrac{C_{2(\rho_{a})}T^{1-\rho_{a}}}{b^{4\rho_{a}-1}}
%\end{align*}
%
%
%
%%%%%%%%%%%%%%%%%%%%%%%%%%%%%%%%%%%%%%%%%%%%%%%%%%%%%%%%%%%%%%%%%%%%%%%%%%%%%%%%%%%%%%%%%%%%%%%%   
%\textbf{Case b3:} \textit{$s^{*}$ is eliminated by some sub-optimal cluster.} 
%
%Let $C_{g}^{'}=\lbrace a_{max_{s_{k}}}\in A^{'}|\forall s_{k}\in S \rbrace$ and $C_{g}^{''}=\lbrace a_{max_{s_{k}}}\in A^{''}|\forall s_{k}\in S \rbrace$. A sub-optimal cluster $s_k$ will eliminate $s^*$ in round $g_*$ only if the cluster elimination condition of Algorithm \ref{alg:eclusucb} holds, which is the following when ${*}\in C_{g_{*}}$:
%\begin{align}
%\hat r_{a_{\max_{s_k}}} - c_{a_{\max_{s_k}}} > \hat{r}^{*}+ c^{*}.
%\label{eq:caseb3-ecluselim}
%\end{align}
%Notice that when ${*}\notin C_{g_{*}}$, since $r_{a_{max_{s_{k}}}}>r^{*}$, the inequality in \eqref{eq:caseb3-ecluselim} has to hold for cluster $s_k$ to eliminate $s^*$.
%As in case $b2$, the probability that a given sub-optimal cluster $s_k$ eliminates $s^*$ is upper bounded by  $\frac{2}{(\psi T\epsilon_{g_{s^{*}}}^{2})^{\rho_{s}}}$ and all sub-optimal clusters with $g_{s_{j}}< g_{*}$ are eliminated before round $g_*$. 
%
%This leaves any arm $a_{\max_{s_{b}}}$ such that $g_{s_{b}}\geq g_{*}$ to still survive and eliminate arm ${*}$ in round $g_{*}$. Let, such arms that survive ${*}$ belong to $C_{g}^{''}$. Hence, following the same way as case $b2$,  the maximal regret after eliminating ${*}$ is,
% \begin{align*}
% \!\!\sum_{g_{*}=0}^{\max\limits_{a_{\max_{s_{j}}}\in C_{g}^{'}}g_{s_{j}}}\!\!\!\!\!\sum_{\substack{\scriptsize a_{\max_{s_{k}}}\in C_{g}^{''}: \\ g_{s_{k}} \geq g_{*}}}\bigg(\dfrac{2}{(\psi T\epsilon_{g_{s^{*}}}^{2})^{\rho_{s}}} \bigg)T\max_{\substack{a_{\max_{s_{j}}}\in C_{g}^{''}: \\ g_{s_{j}}\geq g_{*}}}{\Delta}_{a_{\max_{s_{j}}}}
% \end{align*}
%Using $A'\supset C_{g}^{'}$ and $A''\supset C_{g}^{''}$, we can bound the regret contribution from this case in a similar manner as Case $b2$ as follows:
%
%\begin{align*}
% &\!\!\sum_{i\in A^{'}\setminus A_{s^*}^{'}}\frac{T^{1-\rho_{s}}\rho_{s}^{2\rho_{s}}2^{2\rho_{s}+\frac{3}{2}}}{\psi^{\rho_{s}}\Delta_{i}^{4\rho_{s}-1}} 
% \!+\!\!\!\sum_{i\in A^{''}\setminus A^{'}\cup A_{s^*}^{'}}\!\!\!\!\frac{T^{1-\rho_{s}}\rho_{s}^{2\rho_{s}}2^{2\rho_{s}+\frac{3}{2}}}{\psi^{\rho_{s}}b^{4\rho_{s}-1}} \\
% & = \sum_{i\in A^{'}\setminus A_{s^*}^{'}}\frac{C_{2}(\rho_{s})T^{1-\rho_{s}}}{\Delta_{i}^{4\rho_{s}-1}} +\sum_{i\in A^{''}\setminus A^{'}\cup A_{s^*}^{'}}\frac{C_{2}(\rho_{s})T^{1-\rho_{s}}}{b^{4\rho_{s}-1}} 
%\end{align*}
%
%
%
%%%%%%%%%%%%%%%%%%%%%%%%%%%%%%%%%%%%%%%%%%%%%%%%%%%%%%%%%%%%%%%%%%%%%%%%%%%%%%%%%%%%%%%%%%%%%%%%   
% 
%\textbf{Case b4:} \textit{${*}$ is not in $C_{\max(m_{i},g_{s_{k}})}$, but belongs to $B_{\max(m_{i},g_{s_{k}})}$.}
%
%In this case the optimal arm ${*}\in s^{*}$ is not eliminated, also $s^{*}$ is not eliminated. So, for all sub-optimal arms $i$ in $A_{s^*}^{'}$ which gets eliminated on or before $\max \lbrace m_{i},g_{s_{k}} \rbrace$ will get pulled no less than $z_{i} < \bigg\lceil\dfrac{2\log{(\psi T\epsilon_{m_{i}}^{2})}}{\epsilon_{m_{i}}}\bigg\rceil$ number of times, which leads to the following bound the contribution to the expected regret, as in Case $b1$:
%% 
%% \begin{align*}
%%  &\sum_{i\in A^{'}}\Delta_{i}\bigg\lceil\dfrac{2\log{(\psi T\epsilon_{m_{i}}^{2})}}{\epsilon_{m_{i}}}\bigg\rceil
%% \end{align*}
%% Since $a^{*}$ is definitely in $B_{\max(m_{i},g_{s_{k}})}$ then following the same way as Case $b1$ we can show that can be no worse than,
%\begin{align*}
% &\sum_{i\in A_{s^*}^{'}}\bigg\lbrace \Delta_{i}+\dfrac{32\rho_{a}\log{(\psi T\dfrac{\Delta_{i}^{4}}{16\rho_{a}^{2}})}}{\Delta_{i}} \bigg\rbrace 
%\end{align*} 
%
%For arms $a_i \notin s^*$, the contribution to the regret cannot be greater than that in Case $b3$. So the regret is bounded by,
%
%\begin{align*}
%\sum_{i\in A^{'}\setminus A_{s^*}^{'}}\dfrac{C_{2}(\rho_{s})T^{1-\rho_{s}}}{\Delta_{i}^{4\rho_{s}-1}} +\sum_{i\in A^{''}\setminus A^{'} \cup A_{s^*}^{'}}\dfrac{C_{2}(\rho_{s})T^{1-\rho_{s}}}{b^{4\rho_{s}-1}}
%\end{align*}
%The main claim follows by summing the contributions to the expected regret from each of the cases above.
%\end{proof}

%%%%%%%%%%%%%%%%%%%%%%%%%%%%%%%%%%%%%%%%%%%%%%%%%%%%%%%%%%%%


\section{Proof of Theorem~\ref{Result:Theorem:3}}
\label{App:SR_EClusUCB}
\begin{theorem}
\label{Result:Theorem:3}

For every $0<\eta <1$ and $\gamma > 1$, there exists $\tau$ such that for all $T>\tau$ the simple regret of EClusUCB is upper bounded by,
\begin{align*}
SR_{EClusUCB} & \leq 4\log_{2}\left(\dfrac{14 T}{K}\right)\gamma \sum_{i=1}^{K} \Delta_{i} \exp(-\dfrac{c_{0}\sqrt{e}}{4}) \bigg\lbrace K^{\frac{3}{2} +2\rho_{a}} \bigg(\dfrac{\log (\psi T )}{T^{\frac{3}{2}}(\psi T^2)^{\rho_{a}}}\bigg) \\
%%%%%%%%%%%%%%%%%%%%%%%
& + K^{\frac{3}{2} +2\rho_{s}} \bigg(\dfrac{\log (\psi T )}{T^{\frac{3}{2}}(\psi T^2)^{\rho_{s}}}\bigg) \bigg\rbrace
\end{align*}

with probability at least $1-\eta$, where $c_{0}>0$ is a constant.

\end{theorem}


\begin{proof}
We follow the same steps as in Theorem 2, \cite{liu2016modification}. First we will state the two facts used by this proof.

\begin{enumerate}
\item \emph{Fact 1:} From Theorem \ref{Result:Theorem:1} we know that the probability of elimination of a sub-optimal arm in the $\max(m_{i},g_{s_{k}})$ round is $\bigg(\dfrac{2}{(\psi T\epsilon_{m_{i}}^{2})^{\rho_{a}}}\bigg)$ and of a sub-optimal cluster is $\bigg(\dfrac{2}{(\psi  T\epsilon_{g_{s_{k}}}^{2})^{\rho_{s}}}\bigg)$.
\item \emph{Fact 2:} From \cite{tolpin2012mcts} we know that, for every $0<\eta <1$ and $\gamma > 1$, there exists $\tau$ such that for all $T>\tau$ the probability of a sub-optimal arm $i$ being sampled in the $m$-th round is bounded by $Q_{m}\leq 2\gamma \exp(-c_{m}\dfrac{\sqrt{T}}{2})$, where $c_{m}=\dfrac{c_{0}}{2^{m}}$.
\end{enumerate}

We start with an upper bound on the number of plays $\delta_{\max(m_{i},g_{s_{k}})}$ in the $\max(m_{i},g_{s_{k}})$-th round divided by the total number of plays $T$. We know  from Fact $1$  that the total number of arms surviving in the $\max(m_{i},g_{s_{k}})$-th arm is 

\begin{align*}
|B_{\max(m_{i},g_{s_{k}})}|\leq\bigg(\dfrac{2K}{(\psi T\epsilon_{m_{i}}^{2})^{\rho_{a}}}\bigg) + \bigg(\dfrac{2K}{(\psi  T\epsilon_{g_{s_{k}}}^{2})^{\rho_{s}}}\bigg)
\end{align*}     

Again in EClusUCB, we know that the number of pulls allocated for each surviving arm $i$ in the $\max(m_{i},g_{s_{k}})$-th round is $n_{\max(m_{i},g_{s_{k}})}=\dfrac{2\log (\psi T \epsilon_{\max(m_{i},g_{s_{k}})}^{2})}{\epsilon_{\max(m_{i},g_{s_{k}})}}$. Therefore, the proportion of plays $\delta_{\max(m_{i},g_{s_{k}})}$ in the $\max(m_{i},g_{s_{k}})$-th round can be written as,

\begin{align*}
& \delta_{\max(m_{i},g_{s_{k}})} = \dfrac{(|B_{\max(m_{i},g_{s_{k}})}|.n_{\max(m_{i},g_{s_{k}})})}{T}\\
%%%%%%%%%%%%%%%%%%%%%%%%%
&\leq \bigg(\dfrac{1}{T}.\dfrac{2K}{(\psi T\epsilon_{m_{i}}^{2})^{\rho_{a}}}.\dfrac{2\log (\psi T \epsilon_{m_{i}}^{2})}{\epsilon_{m_{i}}}\bigg) + \bigg(\dfrac{1}{T}.\dfrac{2K}{(\psi  T\epsilon_{g_{s_{k}}}^{2})^{\rho_{s}}}.\dfrac{2\log (\psi T \epsilon_{g_{s_{k}}}^{2})}{\epsilon_{g_{s_{k}}}}\bigg)\\
%%%%%%%%%%%%%%%%
& \leq \bigg(\dfrac{4K\log (\psi T \epsilon_{m_{i}}^{2})}{T\epsilon_{m_{i}}(\psi T\epsilon_{m_{i}}^{2})^{\rho_{a}}}\bigg) + \bigg(\dfrac{4K\log (\psi T \epsilon_{g_{s_{k}}}^{2})}{T\epsilon_{g_{s_{k}}}(\psi  T\epsilon_{g_{s_{k}}}^{2})^{\rho_{s}}}\bigg)
\end{align*}

Now, $\epsilon_{m_{i}}\geq \sqrt{\dfrac{K}{14 T}}$ and $\epsilon_{g_{s_{k}}}\geq \sqrt{\dfrac{K}{14 T}}$ for all rounds $m=0,1,2,...,\big \lfloor \dfrac{1}{2}\log_{2} \dfrac{14 T}{K}\big\rfloor$.

\begin{align*}
 \delta_{\max(m_{i},g_{s_{k}})} & \leq \bigg(\dfrac{4K\log (\psi T \epsilon_{m_{i}}^{2})}{T\epsilon_{m_{i}}(\psi T\epsilon_{m_{i}}^{2})^{\rho_{a}}}\bigg) + \bigg(\dfrac{4K\log (\psi T \epsilon_{g_{s_{k}}}^{2})}{T\epsilon_{g_{s_{k}}}(\psi  T\epsilon_{g_{s_{k}}}^{2})^{\rho_{s}}}\bigg)\\
%%%%%%%%%%%%%%%%%%%%%%%%%
 &\leq \bigg(\dfrac{4K\log (\psi T )}{T\epsilon_{M}(\psi T\epsilon_{M}^{2})^{\rho_{a}}}\bigg) + \bigg(\dfrac{4K\log (\psi T )}{T\epsilon_{M}(\psi  T\epsilon_{M}^{2})^{\rho_{s}}}\bigg)\\
%& \leq \bigg(\dfrac{4Ke^{\frac{1}{2}-\rho_{a}}\log (\psi T )}{T^{\frac{3}{2}}(\psi T^2)^{\rho_{a}}}\bigg) + \bigg(\dfrac{4Ke^{\frac{1}{2}-\rho_{s}}\log (\psi T )}{T^{\frac{3}{2}}(\psi T^2)^{\rho_{s}}}\bigg)
%%%%%%%%%%%%%%%%%%%%%%%%%
& \leq \bigg(\dfrac{4K^{\frac{3}{2} +2\rho_a}\log (\psi T )}{T^{\frac{3}{2}}(\psi T^2)^{\rho_{a}}}\bigg) + \bigg(\dfrac{4K^{\frac{3}{2} +2\rho_s}\log (\psi T )}{T^{\frac{3}{2}}(\psi T^2)^{\rho_{s}}}\bigg)
\end{align*}

Now, applying the bound from Fact 2 and taking into consideration that $c_{m}=\dfrac{c_0}{2^m}$, we can show that the probability of the sub-optimal arm $i$ being pulled is bounded above by,

\begin{align*}
P_{i} = \sum_{m=0}^{M} \delta_{m}.Q_{m} &\leq \sum_{m=0}^{M} \bigg\lbrace\bigg(\dfrac{4K^{\frac{3}{2} +2\rho_a}\log (\psi T )}{T^{\frac{3}{2}}(\psi T^2)^{\rho_{a}}}\bigg) + \bigg(\dfrac{4K^{\frac{3}{2} +2\rho_{s}}\log (\psi T )}{T^{\frac{3}{2}}(\psi T^2)^{\rho_{s}}}\bigg)\bigg\rbrace 2\gamma \exp(-\dfrac{c_{m}\sqrt{T}}{4})\\
%%%%%%%%%%%%%%%%%%%%%%%%%%%%%
& \leq M.\bigg\lbrace\bigg(\dfrac{4K^{\frac{3}{2} +2\rho_a}\log (\psi T )}{T^{\frac{3}{2}}(\psi T^2)^{\rho_{a}}}\bigg) + \bigg(\dfrac{4K^{\frac{3}{2} +2\rho_{s}}\log (\psi T )}{T^{\frac{3}{2}}(\psi T^2)^{\rho_{s}}}\bigg)\bigg\rbrace 2\gamma \exp(-\dfrac{c_{0}\sqrt{T}}{2^{M}.4})\\
%%%%%%%%%%%%%%%%%%%%%%%%%%%%%
& \overset{(a)}{\leq} \log_{2}\dfrac{14T}{K}\gamma \exp(-\dfrac{c_{0}\sqrt{e}}{4})\bigg\lbrace\bigg(\dfrac{4K^{\frac{3}{2} +2\rho_{a}}\log (\psi T )}{T^{\frac{3}{2}}(\psi T^2)^{\rho_{a}}}\bigg) + \bigg(\dfrac{K^{\frac{3}{2} +2\rho_{s}}\log (\psi T )}{T^{\frac{3}{2}}(\psi T^2)^{\rho_{s}}}\bigg)\bigg\rbrace 
\end{align*}
%\text{, for $M=\big \lfloor \dfrac{1}{2}\log_{2} \dfrac{14 T}{K}\big\rfloor$}

Here, we get $(a)$ by substituting $M=\big \lfloor \dfrac{1}{2}\log_{2} \dfrac{14 T}{K}\big\rfloor$.  Hence, the simple regret of EClusUCB is upper bounded by,

\begin{align*}
SR_{EClusUCB} &= \sum_{i=1}^{K} \Delta_{i}. P_{i} \leq \sum_{i=1}^{K} \Delta_{i}. \log_{2}\dfrac{14 T}{K}\gamma \exp(-\dfrac{c_{0}\sqrt{e}}{4})\bigg\lbrace\bigg(\dfrac{4K^{\frac{3}{2} +2\rho_{a}}\log (\psi T )}{T^{\frac{3}{2}}(\psi T^2)^{\rho_{a}}}\bigg) + \bigg(\dfrac{4K^{\frac{3}{2} +2\rho_{s}}\log (\psi T )}{T^{\frac{3}{2}}(\psi T^2)^{\rho_{s}}}\bigg)\bigg\rbrace \\
%%%%%%%%%%%%%%%%%%%%%%%%%%%
&\leq 4\log_{2}\dfrac{14 T}{K}\gamma \sum_{i=1}^{K} \Delta_{i} \exp(-\dfrac{c_{0}\sqrt{e}}{4}) \bigg\lbrace K^{\frac{3}{2} +2\rho_{a}} \bigg(\dfrac{\log (\psi T )}{T^{\frac{3}{2}}(\psi T^2)^{\rho_{a}}}\bigg) + K^{\frac{3}{2} +2\rho_{s}} \bigg(\dfrac{\log (\psi T )}{T^{\frac{3}{2}}(\psi T^2)^{\rho_{s}}}\bigg) \bigg\rbrace
\end{align*}
\end{proof}

%\section{Proof of Corollary 3}
%\label{App:Corollary:3}

\begin{corollary}
\label{Result:Corollary:3}
For $\psi=\dfrac{T}{196\log (K)}$, $\rho_a=\dfrac{1}{2}$ and $\rho_s=\dfrac{1}{2}$, the simple regret of EClusUCB is given by,
\begin{align*}
SR_{EClusUCB} \leq 8 \log_{2}\dfrac{14 T}{K} K^{\frac{5}{2}} \gamma \sum_{i=1}^{K} \Delta_{i} \exp(-\dfrac{c_{0}\sqrt{e}}{4})   \bigg(\dfrac{2 \sqrt{14\log (K)} \log (\dfrac{T}{\sqrt{14\log (K)}} )}{T^{3}}\bigg)
\end{align*}
\end{corollary}

\begin{proof}
Putting $\psi=\dfrac{T}{196 \log (K)}$, $\rho_a=\dfrac{1}{2}$ and $\rho_s=\dfrac{1}{2}$ in the simple regret obtained in Theorem \ref{Result:Theorem:3}, we get
\begin{align*}
SR_{EClusUCB} &\leq 8 \log_{2}\dfrac{14 T}{K} K^{\frac{5}{2}} \gamma \sum_{i=1}^{K} \Delta_{i} \exp(-\dfrac{c_{0}\sqrt{e}}{4})   \bigg(\dfrac{\log (\dfrac{T^{2}}{196 \log (K)} )}{T^{\frac{3}{2}}(\frac{T^3}{196\log (K)})^{\frac{1}{2}}}\bigg)\\
& \leq 8 \log_{2}\dfrac{14 T}{K} K^{\frac{5}{2}} \gamma  \sum_{i=1}^{K} \Delta_{i} \exp(-\dfrac{c_{0}\sqrt{e}}{4}) \bigg(\dfrac{2 \sqrt{14\log (K)} \log (\dfrac{T}{\sqrt{14\log (K)}} )}{T^{3}}\bigg)
\end{align*} 

Thus, we see that the simple regret of EClusUCB decreases at the rate of $O\left( \dfrac{\sqrt{\log K}(\log T)^2}{T^3} \right)$, while the simple regret of CCB decreases at the rate of $O\left( \dfrac{(\log T)^2}{T^4} \right)$. A table comparing the simple regret of CCB and EClusUCB is given in Table \ref{tab:regret-bds1}.

\end{proof}

\begin{table}[ht!]
\caption{Simple regret upper bounds for different bandit algorithms}
\label{tab:regret-bds1}
\begin{center}
\begin{tabular}{|c|c|}
\toprule
Algorithm  & Upper bound \\
\midrule
CCB &$O\left(\log_{2}\left(\dfrac{T}{e}\right) K\gamma\sum_{i=1}^{K}\Delta_{i} \exp(2-\dfrac{c_{0}\sqrt{e}}{4})\dfrac{\log T}{T^4}\right)$ \\\midrule
EClusUCB      &$O\left(  \log_{2}\left( \dfrac{T}{K}\right) K^{\frac{5}{2}} \gamma  \sum_{i=1}^{K} \Delta_{i} \exp(-\dfrac{c_{0}\sqrt{e}}{4}) \bigg(\dfrac{ \sqrt{\log (K)} \log (\dfrac{T}{\sqrt{\log (K)}} )}{T^{3}}\bigg) \right)$\\\bottomrule
\end{tabular}
\end{center}
\end{table}


\section{Adaptive Clustered UCB}
\label{App:AClusUCB}

\begin{algorithm}[th!]
\caption{AClusUCB}
\label{alg:aclusucb}
\begin{algorithmic}
\State {\bf Input:} Time horizon $T$, exploration parameters $\rho_a$, $\rho_s$ and $\psi$.
\State {\bf Initialization:} Set $m:=0$, $B_{0}:=A$, $S_0 = S$, $\epsilon_{0}:=1$, $M=\big \lfloor \dfrac{1}{2}\log_{2} \dfrac{14T}{K}\big\rfloor$, $n_{0}=\bigg\lceil\dfrac{\log{(\psi T\epsilon_{0}^{2})}}{2\epsilon_{0}}\bigg\rceil$, $\ell_{0}:=2$ and  $N_{0}=Kn_{0}$.
%\State Create a partition $S_0$ of the arms into $K$ singleton clusters for each of the $K$ arms.
\State Pull each arm once
\For{$t=K+1,..,T$}	
\State Pull arm $i\in\argmax_{j\in B_{m}}\bigg\lbrace \hat{r}_{j} + \sqrt{\dfrac{\rho_{s}\log{(\psi T\epsilon_{m}^{2})}}{2 z_{j}}} \bigg\rbrace$, where $z_j$ is the number of times arm $j$ has been pulled
\State $t:=t+1$
\State Call CreateClusters()
\ArmElim
\State For each cluster $s_k \in S_{m}$, delete arm ${i}\in s_{k}$ from $B_{m}$ if
\begin{align*}
\hat{r}_{i} + \sqrt{\dfrac{\rho_{a}\log{(\psi T\epsilon_{m}^{2})}}{2 n_{m}}}  < \max_{{j}\in s_{k}}\bigg\lbrace\hat{r}_{j} -\sqrt{\dfrac{\rho_{a}\log{(\psi T\epsilon_{m}^{2})}}{2 n_{m}}} \bigg\rbrace
\end{align*}
% where $\rho_{a}=\dfrac{1}{w_{m}}$ and remove all such arms from $B_{m}$.
\EndArmElim
\ClusElim
\State Delete cluster $s_{k}\in S_{m}$ and remove all arms $i\in s_{k}$ from $B_{m}$ if 
\begin{align*}
 \max_{{i}\in s_{k}}\bigg\lbrace\hat{r}_{i} + \sqrt{\dfrac{\rho_{s}\log{(\psi T\epsilon_{m}^{2})}}{2 n_{m}}}\bigg\rbrace 
 < \max_{{j}\in B_{m}} \bigg\lbrace\hat{r}_{j} - \sqrt{\dfrac{\rho_{s} \log{(\psi T\epsilon_{m}^{2})}}{2 n_{m}}}\bigg\rbrace.
\end{align*}
%  and remove all such arms in the cluster $s_{k}$ from $B_{m}$ to obtain $B_{m+1}$.
\EndClusElim

\If{$t\geq N_{m}$ and $m\leq M$}
\ResParam
\State $\epsilon_{m+1}:=\dfrac{\epsilon_{m}}{2}$\vspace{0.5ex}
\State $\ell_{m+1}:=2\ell_{m}$
\State $B_{m+1}:=B_{m}$
\State $n_{m+1}:=\bigg\lceil\dfrac{\log{(\psi T\epsilon_{m+1}^{2})}}{2\epsilon_{m+1}}\bigg\rceil$
\State $N_{m+1}:=t+|B_{m+1}| n_{m+1}$
\State $m:=m+1$
\EndResParam
\State Stop if $|B_{m}|=1$ and pull ${i}\in B_{m}$ till $T$ is reached.
\EndIf
\EndFor
\Procedure{CreateClusters}{}
\State Create singleton cluster $\lbrace i\rbrace$ for each arm $i\in B_{m}$ and call this partition as $S_{m}$.
\State For two cluster $s_{k},s_{d}\in S_{m}$, join the clusters if any $|\hat{r}_{i}-\hat{r_{j}}| \leq \epsilon_{m}$ and $|s_{k}|+|s_{d}|\leq \ell_{m}$, where $i\in s_{k}$ and $j\in s_{d}$ 
\EndProcedure
\end{algorithmic}
\end{algorithm}

In Section \ref{sec:eclusucb}, we saw that EClusUCB deals with too much early exploration through optimistic greedy sampling. This reduces the cumulative regret, but still one of the principal disadvantages that EClusUCB suffers from is the lack of knowledge of the number of clusters $p$. One way to handle this is to estimate the number of clusters on the fly. In Algorithm \ref{alg:aclusucb}, named Adaptive Clustered UCB , hence referred to as AClusUCB, we explore this idea. AClusUCB uses \emph{hierarchical clustering} (see \citet{friedman2001elements}) to find the number of clusters present. AClusUCB is similar to EClusUCB with two major differences. The first difference is the call to procedure CreateClusters at every timestep. CreateClusters subroutine first creates a singleton cluster for each of the surviving arms in $B_{m}$ and then clusters those singleton clusters $s_{k}, s_{d}\in S_{m}$ (say) into one, if any arm $i\in s_{k}$ and $j\in s_{d}$ is such that $|\hat{r}_{i}-\hat{r}_{j}|\leq \epsilon_{m}$. We cluster based on $\epsilon_{m}$ because we have no prior knowledge of the gaps and we estimate the gap by $\epsilon_{m}$. Also, we destroy the clusters after every timestep and reconstruct the clusters based on the condition specified. Since, the environment is stochastic, the initial clusters will have very poor purity (arms with $\epsilon_{m}$-close expected means lying in a single cluster) whereas in the later rounds the purity becomes better which leads to the optimal arm $*$ lying in a single cluster of its own which will eliminate all the other clusters based on the cluster elimination condition. The second difference is that, we limit the cluster size from start by $\ell_{m}=2$ and then double it after every round. Since the environment is stochastic, if we do not limit the cluster size, then it will result in huge chains of clusters in the initial rounds because the initial estimates of $\hat{r}_{i},\forall i\in A$ will be poor. This condition helps in stopping such large chains of clusters.  

One of the main disadvantages of AClusUCB is that it does not come with a regret upper bound proof. We do not believe that its regret upper bound can be proved in the same way as EClusUCB. The reason for this is that $a_{\max_{s_{k}}}$, the true best arm of a cluster is not fixed in AClusUCB as it deconstructs and then reconstructs the clusters at every timestep. This is not an issue with EClusUCB as it fixes the clusters from beginning and hence $a_{\max_{s_{k}}}$ for each cluster $s_k$ is fixed from the start.

\section{More Experiments}
\label{App:MoreExp}

\begin{figure}[!h]
    \centering
    \begin{tabular}{cc}
    \subfigure[0.32\textwidth][Experiment $5$: Cumulative regret for EClusUCB for various clusters and AClusUCB]
    {
    		\pgfplotsset{
		tick label style={font=\Large},
		label style={font=\Large},
		legend style={font=\Large},
		ylabel style={yshift=32pt},
		}
        \begin{tikzpicture}[scale=0.5]
      	\begin{axis}[
		ylabel={Cumulative Regret},
		xlabel={Clusters},
		grid=major,
        %clip mode=individual,grid,grid style={gray!30},
        clip=true,
        %clip mode=individual,grid,grid style={gray!30},
  		legend style={at={(0.5,1.3)},anchor=north, legend columns=3} ]
      	% UCB
      	%\addplot table{results/NewExpt/Expt4/plotFinalEclUCB01.txt};
		%\addplot table{results/NewExpt/Expt4/plotFinalAclUCB01.txt};
		\addplot table{results/NewExpt/Expt4/plotFinalAEclUCB02.txt};
		\addplot table{results/NewExpt/Expt4/plotFinalFEclUCB02.txt};
      	%\legend{EClusUCBA,AClusUCBA,EClusUCB,AClusUCB} 
      	\legend{AClusUCB,EClusUCB} 
      	\end{axis}
      	\end{tikzpicture}
  		\label{fig:5}
    }
    &
    \subfigure[0.32\textwidth][Experiment $6$: Error Percentage of EClusUCB and CCB]
    {
    		\pgfplotsset{
		tick label style={font=\Large},
		label style={font=\Large},
		legend style={font=\Large},
		ylabel style={yshift=12pt},
		}
        \begin{tikzpicture}[scale=0.5]
      	\begin{axis}[
		xlabel={timestep},
		ylabel={Error Percentage},
		grid=major,
        %clip mode=individual,grid,grid style={gray!30},
        clip=true,
        %clip mode=individual,grid,grid style={gray!30},
  		legend style={at={(0.5,1.3)},anchor=north, legend columns=3} ]
      	% UCB
      	\addplot table{results/NewExpt/Expt5/CCB01_comp_subsampled.txt};
      	\addplot table{results/NewExpt/Expt5/FEclUCB01_comp_subsampled.txt};
      	\legend{CCB,EClusUCB} 
      	\end{axis}
      	\end{tikzpicture}
  		\label{fig:6}
    }
	\end{tabular}
	\label{fig:furtherExpt2}
    \caption{Cumulative regret and Error Percentage for ClusUCB variants}
\end{figure}

\textbf{Fifth experiment:}	This experiment is similar to the testbed in experiment $4$. The experiment is performed over a testbed having $30$ Bernoulli-distributed arms with $r_{i_{:{{i}\neq {*}}}}=0.07,\forall i\in A$ and $r^{*}=0.1$. For each cluster $p=1$ to $\frac{K}{2}$, the cumulative regret of EClusUCB is averaged over 100 independent runs. In Figure \ref{fig:5}, we report the cumulative regret over $T=80000$ timesteps. Here, along with  ECLusUCB we show cumulative regret for AClusUCB (which does not have $p$ as an input parameter) as a straight line, constant over the number of clusters. AClusUCB  performs poorly as compared to EClusUCB for all choices of $p=1$ to $\frac{K}{2}$. We conjecture that this happens because AClusUCB conducts a significant amount of initial exploration to find the number of clusters or till the optimal arm settles in its own cluster which will eliminate all the other clusters as opposed to EClusUCB which has an uniform clustering scheme from the very start. Again note that $p=1$ gives us EClusUCB-AE (EClusUCB with only arm elimination) and it has a matching performance with AClusUCB.  

	
\textbf{Sixth experiment:} This is conducted to analyze the anytime simple regret guarantee of EClusUCB and CCB. The testbed consists of $300$ Gaussian Distributed arms with $r_{i_{:{{i}\neq {*}}}}=0.6,\forall i\in A$, $r^{*}=0.9$ and $\sigma_{i}^{2}=0.5,\forall i\in A$ (similar to the experiment in \cite{liu2016modification}). Each algorithm is run independently $100$ times for $30000$ timesteps and the arm with the maximum $\hat{r}_i,\forall i\in A$ as suggested by the algorithms at every timestep is recorded. The output is considered erroneous if the suggested arm is not the optimal arm. The error percentage over $100$ runs is plotted against $30000$ timesteps and shown in Figure \ref{fig:5}. The exploration regulatory factor for CCB is chosen as $d_i=\frac{\sqrt{T}}{z_{i}}$ (where $z_{i}$ is the number of times an arm ${i}$ has been sampled) as this was found to perform the best in \cite{liu2016modification}. Here we see that the performance of EClusUCB is slightly poorer than CCB towards the end of the horizon as CCB settles for a lower error percentage than EClusUCB.

%	In the fifth experiment, we check the performance of EClusUCB against $4$ algorithms, UCB1, OCUCB\cite{lattimore2015optimally}, MOSS and TS. The testbed is similar to Experiment $1$, with $20$ arms involving Bernoulli reward distribution with expected rewards of the arms $r_{i_{{i}\neq {*}}}=0.07$ and $r^{*}=0.1$ and horizon $T$ is set to $60000$. The number of clusters $p$ is set to $5$ and the result is shown in Figure \ref{fig:5}. Here, we choose the $\psi=\lbrace \frac{T}{\log (KT)}, \log T\rbrace$. From the result we see that EClusUCB is sensitive to the exploratory regulatory factor $\psi$. The first choice of $\psi$ is conservative and does too much exploration while the second choice is optimistic and explores less. Both $\rho_{a}$ and $\rho_{s}$ is set to $\frac{1}{2}$ in all the cases. We see that OCUCB will eventually perform better than MOSS which does not stabilize in $60000$ timesteps but both of them is beaten by EClusUCB. The careful choosing of $\rho_{a}$, $\rho_{s}$ and $\psi$ enables this.
%	
%	In the sixth Experiment the testbed is same as above and we check the performance of EClusUCB with various parameters against MOSS and TS and the result is shown in Figure \ref{fig:6}. In all the cases the number of clusters is set at $5$. Here we see that a lesser exploration regulatory factor $\psi=\log T$ and $\rho_{a}=\rho_{s}=0.25$ outperforms most of the other variants. Infact EClusUCB1, EClusUCB3 and EClusUCB4 are better than MOSS. In both the figures we see that EClusUCB($\log T$) performance is almost similar to AClusUCB.
%
%\begin{figure}[!tbp]
%    \centering
%    \begin{tabular}{cc}
%    \subfigure[0.32\textwidth][Experiment $5$: Cumulative regret for EClusUCB and AClusUCB against various other UCB variants and Thompson Sampling]
%    {
%    		\pgfplotsset{
%		tick label style={font=\Huge},
%		label style={font=\Huge},
%		legend style={font=\Large},
%		ylabel style={yshift=32pt},
%		}
%        \begin{tikzpicture}[scale=0.5]
%      	\begin{axis}[
%		xlabel={timestep},
%		ylabel={Cumulative Regret},
%		grid=major,
%        %clip mode=individual,grid,grid style={gray!30},
%        clip=true,
%        %clip mode=individual,grid,grid style={gray!30},
%  		legend style={at={(0.5,1.5)},anchor=north, legend columns=3} ]
%      	% UCB
%		\addplot table{results/Expt5/CLUCBo1comp_subsampled.txt};
%		\addplot table{results/Expt5/CLUCBo2comp_subsampled.txt};
%		\addplot table{results/Expt5/OCUCBcomp_subsampled.txt};
%		\addplot table{results/Expt5/ACLUCB2comp_subsampled.txt};
%		\addplot table{results/Expt5/UCB1comp_subsampled.txt};
%		\addplot table{results/Expt5/MOSScomp_subsampled.txt};
%		\addplot table{results/Expt5/TScomp_subsampled.txt};
%      	\legend{EClusUCB($\log T$),EClusUCB($\frac{T}{\log(KT)}$),OCUCB,AClusUCB,UCB1,MOSS,TS}
%      	\end{axis}
%      	\end{tikzpicture}
%  		\label{fig:5}
%    }
%    &
%    \subfigure[0.32\textwidth][Experiment $6$: Cumulative regret for EClusUCB with different parameters and AClusUCB. EClusUCB1: $\psi=log T, \rho_{a}=\rho_{s}=0.5$. EClusUCB2: $\psi=\frac{T}{\log(KT)}, \rho_{a}=\rho_{s}=0.5$. EClusUCB3: $\psi=log T, \rho_{a}=\rho_{s}=0.25$. EClusUCB4: $\psi=\frac{T}{\log(KT)}, \rho_{a}=\rho_{s}=0.25$ ]
%    {
%    	\pgfplotsset{
%		tick label style={font=\Huge},
%		label style={font=\Huge},
%		legend style={font=\Large},
%		ylabel style={yshift=32pt},
%		}
%        \begin{tikzpicture}[scale=0.5]
%        \begin{axis}[
%		xlabel={timestep},
%		ylabel={Cumulative Regret},
%        %clip mode=individual,grid,grid style={gray!30},
%		grid=major,
%		clip=true,
%  		legend style={at={(0.5,1.5)},anchor=north, legend columns=3} ]
%        % UCB
%		\addplot table{results/Expt5/CLUCBo1comp_subsampled.txt};
%		\addplot table{results/Expt5/CLUCBo2comp_subsampled.txt};
%		\addplot table{results/Expt5/CLUCB1comp_subsampled.txt};
%		\addplot table{results/Expt5/CLUCB2comp_subsampled.txt};
%		\addplot table{results/Expt5/UCB1comp_subsampled.txt};
%		\addplot table{results/Expt5/MOSScomp_subsampled.txt};
%		\addplot table{results/Expt5/TScomp_subsampled.txt};
%		\addplot table{results/Expt5/ACLUCB2comp_subsampled.txt};
%	 	\legend{EClusUCB1,EClusUCB2,EClusUCB3,EClusUCB4,UCB1,MOSS,TS,AClusUCB}
%      	\end{axis}
%        \end{tikzpicture}
%        \label{fig:6}
%    }
%    \end{tabular}
%   \label{fig:furtherExpt2}
%  	\caption{Cumulative regret for various bandit algorithms on two stochastic K-armed bandit environments. }
%\end{figure}

%\section*{To be commented later, Explaining the working of EClusUCB}
%\begin{table}[!h]
%\begin{center}
%\begin{tabular}{p{1cm}p{1cm}p{1cm}p{1cm}p{1cm}p{1cm}p{1cm}p{7cm}}
%\toprule
%Round  & $a_1$ & $a_2$ & $*$ & $|B_{m}|$ & $n_m$& $|B_{m}|n_m$& Remarks\\
%\hline
%1	& 20 	& 150	& 10		& 3	& 60		&180		& * performs worse\\
%2	& 50		& 230	& 20		& 3	& 100	&300		& Let $m=2$ is the minimum round arm $a_2$ gets eliminated. So for $a_2$ we get $n_{m_2}=300$.  \\
%3	& 500		& 300	& 100		& 3	& 300	&900		& The instant $a_2$ gets pulled $n_{m_2}$ times it gets eliminated as arm elimination is being checked every timestep.\\
%4	& 700		& Elim	& 700		& 2	& 700	& 1400	& Let $m=4$ is the minimum round arm $a_1$ gets eliminated. So for $a_1$ we get $n_{m_1}=700$. And so we are left with only *.\\
%5	& Elim		& Elim	& 900		& 1	& 900	& 900	& * arm outputed.\\
%\end{tabular}
%\end{center}
%\end{table}
  	



\end{document}
